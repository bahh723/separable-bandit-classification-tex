\section{NP-hardness of the weak labeling problem}
\label{section:np-hardness-of-weak-labeling-problem}

Any algorithm for the bandit setting collects information in the form of so
called \emph{strongly labeled} and \emph{weakly labeled} examples.
Strongly-labeled examples are those for which we know the class label. Weakly
labeled example is an example for which we know that class label can be anything
except for a particular one class.

A natural strategy for each round is to find vectors $w_1, w_2, \dots, w_K$ that
linearly separate the examples seen in the previous rounds and use the vectors
to predict the label in the next round. More precisely, we want to find both the
vectors $w_1, w_2, \dots, w_K$ and label for each example consistent with its
weak and/or strong labels such that $w_1, w_2, \dots, w_K$ linearly separate the
labeled examples. We show this problem is NP-hard even for $K=3$.

Clearly, the problem is at least as hard as the decision version of the problem
where the goal is to determine if such vectors and labeling exist. We show that
this problem is NP-complete.

We use symbols $1,2,\dots,K$ for strong labels and symbols $\overline{1},
\overline{2}, \dots, \overline{K}$ for weak labels. Formally, the weak
labeling problem can be described as below:
\begin{figure}[H]
\begin{framed}
\begin{center}
    \textbf{Weak Labeling}
\end{center}
\textbf{Given:} Feature-label pairs $(x_1, y_1)$, $(x_2, y_2)$, \dots, $(x_T, y_T)$ in $\{0,1\}^d \times \{1,2,\dots, K, \overline{1}, \overline{2}, \dots, \overline{K}\}$. \\
\textbf{Question:} Do there exist $ w_1, w_2, \dots, w_K \in \R^d$ such that for all $t=1,2,\dots,T$,
\begin{align*}
& y_t \in \{1,2,\dots,K\} \Longrightarrow \forall i \in \{1,2,\dots,K\} \setminus \{y_t\} \quad \ip{w_{y_t}}{x_t}  > \ip{w_i}{x_t} \; , \\
& \text{and} \\
& y_t \in \{\overline{1}, \overline{2},\dots, \overline{K}\} \Longrightarrow \exists i \in \{1,2,\dots,K\} \quad \ip{w_i}{x_t} > \ip{w_{\overline{y_t}}}{x_t} \; ?
\end{align*}
\end{framed}
\end{figure}

The hardness proof is based on a reduction from the set splitting problem, which
is proven to be NP-complete by Lov\'asz \cite{Garey-Johnson-1979}, to our weak
labeling problem. The reduction is adapted from \cite{Blum-Rivest-1993}.
\begin{figure}[H]
\begin{framed}
\begin{center}
    \textbf{Set Splitting}
\end{center}
\textbf{Given:} A finite set $S$ and a collection $C$ of subsets $c_i$ of $S$. \\
\textbf{Question:} Do there exist disjoint sets $S_1$ and $S_2$ such that $S_1 \cup S_2 = S$ and $\forall i, c_i\not\subseteq S_1$ and $c_i\not\subseteq S_2$?
\end{framed}
\end{figure}

Below we show the reduction. Suppose we are given an instance of the set
splitting problem
\begin{align*}
S = \{1, 2, \dots, N\} \; ,
C = \{c_1, c_2, \dots, c_M\} \; .
\end{align*}

We create the weak labeling instance as follows. Let $d=N+1$ and $K=3$.
Define $\zero$ as the zero vector $(0,\dots,0)\in \R^N$ and $\e_i$ as the
$i$-th standard vector $(0,\dots, 1, \dots, 0)\in \R^N$). Then we include all
the following feature-label pairs:
\begin{itemize}
\item Type 1: $(x,y)=((\zero,1), 3)$,
\item Type 2: $(x,y)=((\e_i,1), \overline{3})$ for all $i \in \{1,2,\dots,N\}$,
\item Type 3: $(x,y)=\left(\left(\sum_{i\in c_j} \e_i, 1\right), 3\right)$ for all $j \in \{1,2,\dots,M\}$.
\end{itemize}

For example, if we have $S=\{1,2,3\}$, $C=\{c_1, c_2\}$, $c_1 = \{1,2\}$,
$c_2=\{2,3\}$, then we create the weak labeling sample set as:
\[
\{
((0,0,0,1),3), ((1,0,0,1),\overline{3}), ((0,1,0,1),\overline{3}),
((0,0,1,1),\overline{3}), ((1,1,0,1),3), ((0,1,1,1),3)
\} \; .
\]
The following lemma shows that answering this weak labeling problem is
equivalent to answering the original set splitting problem.

\begin{lemma}
Any instance of the set splitting problem is a YES instance if and only if the
corresponding instance of the weak labeling problem (as described above) is a
YES instance.
\end{lemma}

\begin{proof}
$(\Longrightarrow)$ Let $S_1, S_2$ be the solution of the set splitting problem. Define
$$
w_1 = \left(a_1, a_2, \cdots, a_N, -\frac{1}{2}\right),
$$
where for all $i \in \{1,2,\dots,N\}$, $a_i=1$ if $i\in S_1$ and $a_i=-N$ if
$i\notin S_1$. Similarly, define
$$
w_2 = \left(b_1, b_2, \cdots, b_N, -\frac{1}{2}\right),
$$
where for all $i \in \{1,2,\dots,N\}$, $b_i=1$ if $i \in S_2$ and $b_i=-N$ if
$i\notin S_2$. Finally, define
$$
w_3 = (0,0,\cdots, 0),
$$
the zero vector. To see this is a solution for the weak labeling problem, we
verify separately for Type 1-3 samples defined above. For Type 1 sample, we have
$$
\ip{w_3}{x} = 0 > -\frac{1}{2} = \ip{w_1}{x}=\ip{w_2}{x}.
$$
For a Type 2 sample that corresponds to index $i$, we have either $i\in S_1$ or
$i\in S_2$ because $S_1\cup S_2 = \{1,2,\dots,N\}$ is guaranteed. Thus, either
$a_i=1$ or $b_i=1$. If $a_i=1$ is the case, then
$$
\ip{w_1}{x} = a_i - \frac{1}{2} = \frac{1}{2} > 0 = \ip{w_3}{x};
$$
similarly if $b_i=1$, we have $\ip{w_2}{x}>\ip{w_3}{x}$. \\ For a Type 3 sample
that corresponds to index $j$, Since $c_j \not\subset S_1$, there exists some
$i'\in c_j$ and $i'\notin S_1$. Thus we have $x_{i'}=1$, $a_{i'}=-N$, and
therefore
\begin{align*}
\ip{w_1}{x}
& = a_{i'}x_{i'} + \sum_{i\in \{1,2,\dots,N\} \setminus \{i'\}} a_ix_i - \frac{1}{2} \\
& \le -N + (N-1)-\frac{1}{2} < 0 = \ip{w_3}{x} \; .
\end{align*}
Because $c_j \not\subset S_2$ also holds, we also have
$\ip{w_2}{x}<\ip{w_3}{x}$. This direction is therefore proved. \\
\ \\
$(\Longleftarrow)$ Given the solution $w_1, w_2, w_3$ of the weak labeling problem, we define
\begin{align*}
S_1 &= \left\{i \in \{1,2,\dots,n\} ~:~ \ip{w_1-w_3}{(\e_i, 1)} > 0 \right\}, \\
S_2 &= \left\{i \in \{1,2,\dots,n\} ~:~ \ip{w_2-w_3}{(\e_i, 1)} > 0 \text{\ and\ } i \notin S_1 \right\}.
\end{align*}
It is not hard to see $S_1 \cap S_2 = \emptyset$ and $S_1\cup S_2 =
\{1,2,\dots,N\}$. The former is because $S_2$ only includes elements that are
not in $S_1$. For the latter, note that $(\e_i, 1)$ is the feature vector for
Type 2 samples. Because Type 2 samples all have label $\overline{3}$, for any $i
\in \{1,2,\dots,N\}$, one of the following must hold: $\ip{w_1-w_3}{(\e_i,
1)}>0$ or $\ip{w_2-w_3}{(\e_i, 1)}>0$. This implies $i\in S_1$ or $i\in S_2$.

Now we show $\forall j$, $c_j \not\subset S_1$ and $c_j \not\subset S_2$ by
contradiction. Assume there exists some $j$ such that $c_j \subset S_1$. By our
definition of $S_1$, we have $\ip{w_1-w_3}{(\e_i, 1)} > 0$ for all $i\in c_j$.
Therefore,
\begin{align*}
\sum_{i\in c_j} \ip{w_1-w_3}{\left(\e_i, 1\right)}
 = \ip{w_1-w_3}{\left(\sum_{i\in c_j} \e_i, |c_j|\right)} > 0.
\end{align*}
Because Type 1 sample has label $3$, we also have
$$
\ip{w_1-w_3}{\left(\zero, 1\right)} < 0.
$$
Combining the above two inequalities, we get
\begin{align*}
& \ip{w_1-w_3}{\left(\sum_{i\in c_j}\e_i, 1\right)}
 = \ip{w_1-w_3}{\left(\sum_{i\in c_j}\e_i, |c_j|\right)} - (|c_j|-1)\ip{w_1-w_3}{\left(\zero, 1\right)}  > 0 \; .
\end{align*}
Note that $\left(\sum_{i\in c_j}\e_i, 1\right)$ is a feature vector for Type 3
samples. Thus the above inequality contradicts that Type 3 samples have label 3.
Therefore, $c_j \not\subset S_1$. If we assume there exists some $c_j \subset
S_2$, same arguments apply and also lead to contradiction.
\end{proof}
