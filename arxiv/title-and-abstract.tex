% The \icmltitle you define below is probably too long as a header.
% Therefore, a short form for the running title is supplied here:
%\icmltitlerunning{Bandit Multiclass Linear Classification: Efficient Algorithms for the Separable Case}

%\twocolumn[
%\icmltitle{Bandit Multiclass Linear Classification: \\ Efficient Algorithms for the Separable Case}

\title{Bandit Multiclass Linear Classification: \\ Efficient Algorithms for the Separable Case}

%\begin{icmlauthorlist}
\author[1]{Alina Beygelzimer}
\author[1]{D\'avid P\'al}
\author[1]{Bal\'azs Sz\"or\'enyi}
\author[2]{Devanathan Thiruvenkatachari}
\author[3]{Chen-Yu Wei}
\author[4]{Chicheng Zhang}

%\end{icmlauthorlist}

\affil[1]{Yahoo Research, New York, NY, USA}
\affil[2]{New York University, New York, NY, USA}
\affil[3]{University of Southern California, Los Angeles, CA, USA}
\affil[4]{Microsoft Research, New York, NY, USA}

\maketitle
%\icmlcorrespondingauthor{D\'avid P\'al}{davidko.pal@gmail.com}

%\icmlkeywords{multi-armed bandits, contextual bandits, online classification, linear separability}

%\vskip 0.3in
%]

%\printAffiliationsAndNotice{}

\begin{abstract}
We study the problem of efficient online multiclass linear classification with
bandit feedback, where all examples belong to one of $K$ classes and lie in the
$d$-dimensional Euclidean space. Previous works have left open the challenge of
designing efficient algorithms with finite mistake bounds when the data is
linearly separable by a margin $\gamma$. In this work, we take a first step
towards this problem. We consider two notions of linear separability,
\emph{strong} and \emph{weak}.

\begin{enumerate}
\item Under the strong linear separability condition, we design an efficient
algorithm that achieves a near-optimal mistake bound of
$O\left( K/\gamma^2 \right)$.

\item Under the more challenging weak linear separability condition, we design
an efficient algorithm with a mistake bound of $\min (2^{\widetilde{O}(K \log^2
(1/\gamma))}, 2^{\widetilde{O}(\sqrt{1/\gamma} \log K)})$ \footnote{We use the notation
$\widetilde{O}(f(\cdot)) = O(f(\cdot) \polylog(f(\cdot)))$.}.
Our algorithm
is based on kernel Perceptron, which is inspired by the work
of \citet{Klivans-Servedio-2008} on improperly learning intersection of halfspaces.
%is based on a infinite-dimensional feature mapping that transforms weakly linear
%separable examples to strongly linear separable examples, which in turn is
%of \citet{Shalev-Shwartz-Shamir-Sridharan-2011},
\end{enumerate}
\end{abstract}
