\section{Proof of Lemma~\ref{lemma:norm-bound}}
\label{section:proof-norm-bound}

\begin{proof}
Note that the polynomial $p$ can be written as
$p(x) = \sum_{\alpha_1, \alpha_2, \dots, \alpha_d} c'_{\alpha_1, \alpha_2, \dots, \alpha_d} x_1^{\alpha_1} x_2^{\alpha_2} \dots x_d^{\alpha_d}$.
We define $c \in \ell_2$ using the multi-index notation as
$$
c_{\alpha_1, \alpha_2, \dots, \alpha_d}
= \frac{c'_{\alpha_1, \alpha_2, \dots, \alpha_d} 2^{(\alpha_1 + \alpha_2 + \dots + \alpha_d)/2}}{\sqrt{\binom{\alpha_1 + \alpha_2 + \dots + \alpha_d}{\alpha_1, \alpha_2, \dots, \alpha_d}}}
$$
for all tuples $(\alpha_1, \alpha_2, \dots, \alpha_d)$ such that $\alpha_1 + \alpha_2 + \dots + \alpha_d \le \deg(p)$.
Otherwise, we define $c_{\alpha_1, \alpha_2, \dots, \alpha_d} = 0$. By the definition
of $\phi$, $\ip{c}{\phi(x)}_{\ell_2} = p(x)$.

Whether $\alpha_1 + \ldots + \alpha_d \leq \deg(p)$, we always have:
\begin{align*}
|c_{\alpha_1, \alpha_2, \dots, \alpha_d}|
 \le 2^{(\alpha_1 + \alpha_2 + \dots + \alpha_d)/2} |c'_{\alpha_1, \alpha_2, \dots, \alpha_d}|
 \le 2^{\deg(p)/2} |c'_{\alpha_1, \alpha_2, \dots, \alpha_d}| \; .
\end{align*}
Therefore,
\begin{align*}
\norm{c}_{\ell_2}
 \le 2^{\deg(p)/2} \sqrt{\sum_{\alpha_1, \alpha_2, \dots, \alpha_d} (c'_{\alpha_1, \alpha_2, \dots, \alpha_d})^2} 
 = 2^{\deg(p)/2} \norm{p} \; . \qquad \qedhere
\end{align*}
\end{proof}
