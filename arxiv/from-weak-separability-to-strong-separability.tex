\section{From weak separability to strong separability}
\label{section:from-weak-separability-to-strong-separability}

In this section, we consider the case when the examples are weakly linearly
separable. Throughout this section, we assume without loss of generality that
all examples lie in the unit ball $\B(\zero,1) \subseteq \R^d$.\footnote{Instead of
working with feature vector $x_t$ we can work with normalized feature vectors
$\widehat{x}_t = \frac{x_t}{\norm{x_t}}$. It can be easily checked that if
$(x_1,y_1), (x_2,y_2), \dots, (x_T,y_T)$ are weakly linearly separable with
margin $\gamma$ and $\norm{x_t} \le R$ for all $t$, then the normalized examples
$(\widehat{x}_1,y_1), (\widehat{x}_2,y_2), \dots, (\widehat{x}_T,y_T)$ are weakly
linearly separable with margin $\gamma/R$.} Note that
Algorithm~\ref{algorithm:algorithm-for-strongly-linearly-separable-examples}
alone does not guarantee a finite mistake bound in this setting, as weak linear
separability does not imply strong linear separability.

We use a positive definite kernel function $k(\cdot, \cdot)$, namely
\emph{rational kernel}~\citep{Shalev-Shwartz-Shamir-Sridharan-2011}, whose
corresponding feature map $\phi(\cdot)$ transforms any sequence of \emph{weakly}
linearly separable examples to \emph{strongly} linearly separable sequence of
examples.
Specifically, $\phi$ has the property that if a set of labeled
examples in $\B(\zero,1)$ is weakly linearly separable with a margin $\gamma$, then
after applying $\phi$ the examples become strongly linearly separable with a
margin $\gamma'$ and their squared norms are bounded by $2$.
\footnote{Other kernels, such as the polynomial kernel
$k(x,x') = (1+\ip{x}{x'})^d$, or the multinomial kernel~\citep{Goel-Klivans-2017} $k(x,x') = \sum_{i=0}^d(\ip{x}{x'})^i$,
will have similar properties for large enough $d$.}
The parameter $\gamma'$ is
a function of the old margin $\gamma$ and the number of classes $K$, and is
specified in \autoref{theorem:margin-transformation} below.

The rational kernel $k:\B(\zero,1) \times \B(\zero,1) \to \R$ is defined as
\begin{equation}
\label{equation:rational-kernel}
k(x,x') = \frac{1}{1 - \frac{1}{2}\ip{x}{x'}_{\R^d}} \; .
\end{equation}
Note that $k(x,x')$ can be evaluated in $O(d)$ time.

Consider the classical real separable Hilbert space $\ell_2 = \{ x \in \R^\infty
~:~ \sum_{i=1}^\infty x_i^2 < + \infty \}$ equipped with the standard inner
product $\ip{x}{x'}_{\ell_2} = \sum_{i=1}^\infty x_i x'_i$. If we index the
coordinates of $\ell_2$ by $d$-tuples $(\alpha_1, \alpha_2, \dots, \alpha_d)$ of
non-negative integers, the feature map that corresponds to $k$ is $\phi: \B(\zero,1)
\to \ell_2$,
\begin{align}
& \left(\phi(x_1, x_2, \dots, x_d)\right)_{(\alpha_1, \alpha_2, \dots, \alpha_d)} = x_1^{\alpha_1} x_2^{\alpha_2} \dots x_d^{\alpha_d} \cdot \sqrt{2^{-(\alpha_1 + \alpha_2 + \dots + \alpha_d)} \binom{\alpha_1 + \alpha_2 + \dots + \alpha_d}{\alpha_1, \alpha_2, \dots, \alpha_d}}
\label{equation:phi}
\end{align}
where $\binom{\alpha_1 + \alpha_2 + \dots + \alpha_d}{\alpha_1, \alpha_2, \dots,
\alpha_d} = \frac{(\alpha_1 + \alpha_2 + \dots + \alpha_d)!}{\alpha_1! \alpha_2!
\dots \alpha_d!}$ is the multinomial coefficient. It can be easily checked that
$$
k(x,x') = \ip{\phi(x)}{\phi(x')}_{\ell_2}.
$$
The last equality together with the formula for $k$ implies that $k(x,x) <
+\infty$ for any $x$ in $\B(\zero,1)$ and thus in particular implies that $\phi(x)$
indeed lies in $\ell_2$.

The following theorem is our main technical result in this section. We defer its
proof to Section~\ref{section:margin-transformation}.

\begin{theorem}[Margin transformation]
\label{theorem:margin-transformation}
Let $(x_1, y_1)$, $(x_2, y_2)$, $\dots$, $(x_T, y_T)$ in $\B(\zero,1) \times
\{1,2,\dots,K\}$ be a sequence of labeled examples that is weakly linearly
separable with margin $\gamma > 0$. Let $\phi$ be as defined in
equation~\eqref{equation:phi} and let
%\begingroup
%\allowdisplaybreaks
%\begin{align*}
%r & = 2 \left\lceil \frac{1}{4} \log_2(4K-3) \right\rceil + 1, \quad \quad s = \left \lceil \log_2(2/\gamma) \right \rceil, \\
%\gamma_1 & = \frac{1}{2\sqrt{K}}  \\
%& \ \cdot \left(376 \lceil \log_2(2K-2) \rceil \cdot \left \lceil \sqrt{\frac{2}{\gamma}} \right \rceil \right)^{-\frac{1}{2} \lceil \log_2(2K-2) \rceil \cdot \left \lceil \sqrt{\frac{2}{\gamma}} \right \rceil}, \\
%\gamma_2 & = \frac{2^{s(s+1)r(K-1)} }{4\sqrt{K}(4K-5) 2^{K-1}} \\
%& \ \cdot \left(2^{s+1} r(K-1) (4s+2)^2 \right)^{-(s+1/2)r(K-1)} \; .
%\end{align*}
%\endgroup
\begingroup
\allowdisplaybreaks
\begin{align*}
%r & = 2 \left\lceil \frac{1}{4} \log_2(4K-3) \right\rceil + 1, \quad \quad s = \left \lceil \log_2(2/\gamma) \right \rceil, \\
\gamma_1 = & \frac{
  \left[
    376 \lceil \log_2(2K-2) \rceil \cdot \left \lceil \sqrt{\frac{2}{\gamma}} \right \rceil
  \right]^{
    \frac{-\lceil \log_2(2K-2) \rceil \cdot \left \lceil \sqrt{{2}/{\gamma}} \right \rceil}{2}
  }
}{2\sqrt{K}},
\gamma_2 = & \frac{
   \left(2^{s+1} r(K-1) (4s+2) \right)^{-(s+1/2)r(K-1)}
}{4\sqrt{K}(4K-5) 2^{K-1}}
\; ,
\end{align*}
\endgroup

%\\
%&  \cdot 2^{s(s+1)r(K-1)}

where $r = 2 \left\lceil \frac{1}{4} \log_2(4K-3) \right\rceil + 1$ and $s = \left \lceil \log_2(2/\gamma) \right \rceil$.
Then, the sequence of labeled examples transformed by $\phi$,
namely $(\phi(x_1), y_1), (\phi(x_2), y_2), \dots,
(\phi(x_T), y_T)$, is strongly linearly separable with margin $\gamma' =
\max\{\gamma_1, \gamma_2\}$. In addition, for all $t$ in $\cbr{1,\ldots,T}$,
$k(x_t,x_t) \leq 2$.
\end{theorem}

Using this theorem we derive a mistake bound for
Algorithm~\ref{algorithm:kernelized} with kernel \eqref{equation:rational-kernel}
under the weak linear separability assumption.

\begin{corollary}[Mistake upper bound]
\label{corollary:weakly-separable-examples-mistake-upper-bound}
Let $K$ be a positive integer and let $\gamma$ be a positive real number. If
$(x_1, y_1), (x_2, y_2), \dots, (x_T, y_T) \in \B(\zero,1) \times \{1,2,\dots,K\}$
is a sequence of weakly separable labeled examples with margin $\gamma > 0$,
then the expected number of mistakes made by Algorithm~\ref{algorithm:kernelized}
with kernel $k(x,x')$ defined by \eqref{equation:rational-kernel}
is at most $\min (2^{\widetilde{O}(K \log^2
(1/\gamma))}, 2^{\widetilde{O}(\sqrt{1/\gamma} \log K)})$.
\end{corollary}

This corollary follows directly from
Theorems~\ref{theorem:kernelized-upper-bound}
and~\ref{theorem:margin-transformation}.

\subsection{Proof of Theorem~\ref{theorem:margin-transformation}}
\label{section:margin-transformation}

\paragraph{Overview.} The idea behind the construction and analysis of the
mapping $\phi$ is polynomial approximation. Specifically, we construct $K$
multivariate polynomials $p_1, p_2, \dots,p_K$ such that
\begin{align}
\label{equation:poly-pos}
& \forall t \in \cbr{1,2,\dots,T}, \qquad p_{y_t}(x_t) \ge \frac{\gamma'}{2} \; ,
\\
\label{equation:poly-neg}
& \begin{gathered}
\forall t \in \cbr{1,2,\dots,T} \ \forall i \in \cbr{1,2,\ldots,K} \setminus \cbr{y_t}, \qquad
p_i(x_t) \le - \frac{\gamma'}{2} \; .
\end{gathered}
\end{align}
We then show (\autoref{lemma:norm-bound}) that each polynomial $p_i$ can be
expressed as $\ip{c_i}{\phi(x)}_{\ell_2}$ for some $c_i \in \ell_2$. This
immediately implies that that the examples $(\phi(x_1),y_1), \ldots,
(\phi(x_T),y_T)$ are strongly linearly separable with a positive margin.

The conditions \eqref{equation:poly-pos} and~\eqref{equation:poly-neg} are
equivalent to that
\begin{align}
\label{equation:poly-pos-i}
\forall t \in \cbr{1,2,\dots,T}, y_t = i \quad \Rightarrow \quad p_i(x_t) \ge \frac{\gamma'}{2} \; , \\
\label{equation:poly-neg-i}
\forall t \in \cbr{1,2,\dots,T}, y_t \neq i \quad \Rightarrow \quad p_i(x_t) \le -\frac{\gamma'}{2} \; .
\end{align}
hold for all $i \in \{1,2,\dots,K\}$. We can thus fix $i$ and focus on
construction of one particular polynomial $p_i$.

Since examples $(x_1,y_1), (x_2, y_2), \dots, (x_T,y_T)$ are weakly linearly separable,
all examples from class $i$ lie in
$$
R_i^+ = \!\!\!\!\! \!\!\!\!\! \bigcap_{j \in \cbr{1,2,\dots,K} \setminus \cbr{i}} \cbr{x \in \B(\zero,1) ~:~ \ip{w_i^* - w_j^*}{x} \ge \gamma},
$$
and all examples from the remaining classes lie in
$$
R_i^- = \!\!\!\!\! \!\!\!\!\! \bigcup_{j \in \cbr{1,2,\dots,K} \setminus \cbr{i}} \cbr{x \in \B(\zero,1) ~:~ \ip{w_i^* - w_j^*}{x} \le -\gamma}.
$$
Therefore, to satisfy conditions~\eqref{equation:poly-pos-i}
and~\eqref{equation:poly-neg-i}, it suffices to construct $p_i$ such that
\begin{align}
\label{eqn:r-plus}
x \in R_i^+ \qquad & \Longrightarrow \qquad p_i(x) \ge \frac {\gamma'} 2 \; , \\
\label{eqn:r-minus}
x \in R_i^- \qquad & \Longrightarrow \qquad p_i(x) \le - \frac {\gamma'} 2 \; .
\end{align}

According to the well known Stone-Weierstrass theorem~\citep[see
e.g.][Section~10.10]{Davidson-Donsig-2010}, on a compact set, multivariate
polynomials uniformly approximate any continuous function. Roughly speaking, the
conditions \eqref{eqn:r-plus} and \eqref{eqn:r-minus} mean that $p_i$
approximates on $\B(\zero,1)$ a scalar multiple of the indicator function of the
intersection of $K-1$ halfspaces $\bigcap_{j \in \cbr{1,2,\dots,K} \setminus
\cbr{i}} \cbr{x: \ip{w_i^* - w_j^*}{x} \geq 0}$ while within margin $\gamma$ along
the decision boundary, the polynomial is allowed to attain arbitrary values.
It is thus clear such a polynomial exists.

We give two explicit constructions for such polynomial in
Theorems~\ref{theorem:polynomial-approximation-1}~and~\ref{theorem:polynomial-approximation-2}.
Our constructions are based on \citet{Klivans-Servedio-2008} which in turn uses
the constructions from~\citet{Beigel-Reingold-Spielman-1995}. More importantly,
the theorems quantify certain parameters of the polynomial, which allows us
to upper bound the transformed margin $\gamma'$.

Before we state the theorems, recall that a polynomial of $d$ variables is a
function $p:\R^d \to \R$ of the form
\begin{align*}
p(x)
 = p(x_1, x_2, \dots, x_d) = \sum_{\alpha_1, \alpha_2, \dots, \alpha_d} c_{\alpha_1, \alpha_2, \dots, \alpha_d} x_1^{\alpha_1} x_2^{\alpha_2} \dots x_d^{\alpha_d}
\end{align*}
where the sum ranges over a finite set of $d$-tuples $(\alpha_1, \alpha_2,
\dots, \alpha_d)$ of non-negative integers and $c_{\alpha_1, \alpha_2, \dots,
\alpha_d}$'s are real coefficients. The \emph{degree} of a polynomial $p$,
denoted by $\deg(p)$, is the largest value of $\alpha_1 + \alpha_2 + \dots +
\alpha_d$ for which the coefficient $c_{\alpha_1, \alpha_2, \dots, \alpha_d}$ is
non-zero. Following the terminology of~\citet{Klivans-Servedio-2008}, the
\emph{norm of a polynomial} $p$ is defined as
$$
\norm{p} = \sqrt{\sum_{\alpha_1, \alpha_2, \dots, \alpha_d} \left(c_{\alpha_1, \alpha_2, \dots, \alpha_d} \right)^2 } \; .
$$
It is easy see that this is indeed a norm, since we can interpret it as the
Euclidean norm of the vector of the coefficients of the polynomial.

\begin{theorem}[Polynomial approximation of intersection of halfspaces I]
\label{theorem:polynomial-approximation-1}
Let $v_1, v_2, \dots, v_m \in \R^d$ be vectors such that $\norm{v_1},
\norm{v_2}, \dots, \norm{v_m} \le 1$. Let $\gamma \in (0,1)$. There exists a
multivariate polynomial $p:\R^d \to \R$ such that
\begin{enumerate}
\item $p(x) \ge 1/2$ for all $\displaystyle x \in R^+ = \bigcap_{i=1}^m \left\{ x \in \B(\zero,1) ~:~  \ip{v_i}{x} \ge \gamma \right\}$,
\item $p(x) \le -1/2$ for all $\displaystyle x \in R^- = \bigcup_{i=1}^m \left\{ x \in \B(\zero,1) ~:~  \ip{v_i}{x} \le - \gamma \right\}$,
\item $\displaystyle \deg(p) = \left\lceil \log_2(2m) \right\rceil \cdot \left\lceil \sqrt{{1}/{\gamma}} \right\rceil$,
\item $\displaystyle \norm{p} \le \left[ 188 \left\lceil \log_2(2m) \right\rceil \cdot \left\lceil \sqrt{{1}/{\gamma}} \right\rceil \right]^{
  \frac{
    \left\lceil \log_2(2m) \right\rceil \cdot \left\lceil \sqrt{{1}/{\gamma}} \right\rceil
  }{2}
}$.
\end{enumerate}
\end{theorem}

\begin{theorem}[Polynomial approximation of intersection of halfspaces II]
\label{theorem:polynomial-approximation-2}
Let $v_1, v_2, \dots, v_m \in \R^d$ be vectors such that $\norm{v_1},
\norm{v_2}, \dots, \norm{v_m} \le 1$. Let $\gamma \in (0,1)$.
Define
$$
r = 2 \left\lceil \frac{1}{4} \log_2(4m + 1) \right\rceil + 1 \quad \text{and} \quad s = \left \lceil \log_2(1/\gamma) \right \rceil \; .
$$
Then, there exists a multivariate polynomial $p:\R^d \to \R$ such that
\begin{enumerate}
\item $\displaystyle p(x) \ge 1/2$
for all $\displaystyle x \in R^+ = \bigcap_{i=1}^m \left\{ x \in \B(\zero,1) ~:~ \ip{v_i}{x} \ge \gamma \right\}$,

\item $\displaystyle p(x) \le - 1/2$
for all $\displaystyle x \in R^- = \bigcup_{i=1}^m \left\{ x \in \B(\zero,1) ~:~ \ip{v_i}{x} \le - \gamma \right\}$,

\item $\deg(p) \le (2s+1) rm$,
\item $\norm{p} \le (4m-1) 2^m \cdot \left(2^s rm (4s+2) \right)^{(s+1/2)rm}$.
\end{enumerate}
\end{theorem}

The proofs of the theorems are in
Appendix~\ref{section:proof-of-polynomial-approximation}. The geometric
interpretation of the two regions $R^+$ and $R^-$ in the theorems is explained
in Figure~\ref{figure:pizza-slice}. Similar but weaker results were proved
by~\citet{Klivans-Servedio-2008}. Specifically, our bounds in parts 1, 2, 3, 4
of
Theorems~\ref{theorem:polynomial-approximation-1}~and~\ref{theorem:polynomial-approximation-2}
are independent of the dimension $d$.

\begin{figure}
\begin{center}
 \scalebox{.65}{\begin{tikzpicture}

  \useasboundingbox (-5.5,-5.5) rectangle (5.5,5.5);
  % \draw[help lines] (-5.5,-5.5) rectangle (5.5,5.5);

  % Grid and coordinate axes
  %% \draw[help lines] (-5.2,-5.2) grid (5.2,5.2);
  %% \draw[thick, ->, >=latex] (-5.2,0) -- (5.2,0);
  %% \draw[thick, ->,  >=latex] (0,-5.2) -- (0,5.2);

  % Notable points
  \coordinate (Origin) at (0,0);

  \coordinate [label={[xshift=-10mm, yshift=5mm]$\ip{v_1}{x} = 0$}] (A) at (120:5);
  \coordinate [label={[xshift=-10mm, yshift=-10mm]$\ip{v_2}{x} = 0$}] (B) at (240:5);

  % Circle which contains all the examples.
  % \draw (Origin) circle (5cm);

  \draw[dashed] (Origin) -- (A);
  \draw[dashed] (Origin) -- (B);

  \coordinate (A1) at ($(120:0.5*sqrt{97} - 0.5) - (1,0)$);
  \coordinate (B1) at ($(240:0.5*sqrt{97} - 0.5) - (1,0)$);

  \coordinate (A2) at ($(120:0.5*sqrt{97} + 0.5) + (1,0)$);
  \coordinate (B2) at ($(240:0.5*sqrt{97} + 0.5) + (1,0)$);

  \draw[ultra thick] ($(Origin) - (1,0)$) -- (A1);
  \draw[ultra thick] ($(Origin) - (1,0)$) -- (B1);

  \draw[ultra thick] ($(Origin) + (1,0)$) -- (A2);
  \draw[ultra thick] ($(Origin) + (1,0)$) -- (B2);

  \pic [ultra thick, draw, angle radius=5cm] {angle=A1--Origin--B1};
  \pic [ultra thick, draw, angle radius=5cm] {angle=B2--Origin--A2};


  \coordinate (C) at (120:3);
  \coordinate [label={[xshift=-6mm, yshift=-4mm]$\gamma$}] (C1) at ($(120:3) + (0.75, 0.43301270189221932338)$);
  \coordinate [label={[xshift=+3mm, yshift=+2mm]$\gamma$}] (C2) at ($(120:3) - (0.75, 0.43301270189221932338)$);
  \draw[<->, >=latex, ultra thick] (C1) -- (C) -- (C2);

  \coordinate (D) at (240:3);
  \coordinate [label={[xshift=+4mm, yshift=-8mm]$\gamma$}] (D1) at ($(240:3) + (-0.75, 0.43301270189221932338)$);
  \coordinate [label={[xshift=-5mm, yshift=-3mm]$\gamma$}] (D2) at ($(240:3) - (-0.75, 0.43301270189221932338)$);
  \draw[<->, >=latex, ultra thick] (D1) -- (D) -- (D2);

  \coordinate [label={$R^+$}](R1) at ($(Origin) - (3,0)$);
  \coordinate [label={$R^-$}](R2) at ($(Origin) + (3,0)$);

\end{tikzpicture}
}
\end{center}
\caption[]{The figure shows the two regions $R^+$ and $R^-$ used in parts 1 and
2 of
Theorems~\ref{theorem:polynomial-approximation-1}~and~\ref{theorem:polynomial-approximation-2}
for the case $m=d=2$ and a particular choice of vectors $v_1, v_2$ and margin
parameter $\gamma$. The separating hyperplanes $\ip{v_1}{x} = 0$ and
$\ip{v_2}{x} = 0$ are shown as dashed lines.}
\label{figure:pizza-slice}
\end{figure}

The following lemma establishes a correspondence between any multivariate
polynomial in $\R^d$ and an element in $\ell_2$, and gives an upper bound on its
norm. Its proof follows from simple algebra, which we defer to
Appendix~\ref{section:proof-norm-bound}.

\begin{lemma}[Norm bound]
\label{lemma:norm-bound}
Let $p:\R^d \to \R$ be a multivariate polynomial.
There exists $c \in \ell_2$ such that $p(x) = \ip{c}{\phi(x)}_{\ell_2}$
and $\norm{c}_{\ell_2} \le 2^{\deg(p)/2} \norm{p}$.
\end{lemma}

Using the lemma and the polynomial approximation theorems, we can prove that the
mapping $\phi$ maps any set of weakly linearly separable examples to a strongly
linearly separable set of examples. Due to space constraints, we defer the full
proof of Theorem~\ref{theorem:margin-transformation} to
Appendix~\ref{section:proof-of-theorem-margin-transformation}.

%\begin{proof}[Proof sketch of Theorem~\ref{theorem:margin-transformation}]
%Since the examples $(x_1, y_1)$,  $\dots$, $(x_T, y_T)$ are weakly
%linearly separable with margin $\gamma$, there are vectors $w_1, w_2, \dots, w_K$
%satisfying \eqref{equation:weak-linear-separability-1} and
%\eqref{equation:weak-linear-separability-2}.

%Fix any $i \in \{1,2,\dots,K\}$. Consider the $K-1$ vectors $(w_i - w_j)/2$ for
%$j \in \{1,2,\dots,K\} \setminus \{i\}$. Note that the vectors have norm at most
%$1$. We consider two cases regarding the relationship between $\gamma_1$ and
%$\gamma_2$.

%\paragraph{Case 1: $\gamma_1 \geq \gamma_2$.} In this setting,
%Theorem~\ref{theorem:polynomial-approximation-1} implies that there exist a
%multivariate polynomial $p_i:\R^d \to \R$

%\begin{align*}
%\deg(p_i) & = \lceil \log_2(2K-2) \rceil \cdot \left\lceil \sqrt{{2}/{\gamma}} \right\rceil \; ,
%\end{align*}
%such that all examples $x$ in $R_i^+$ (resp. $R_i^-$) satisfy $p_i(x) \geq 1/2$
%(resp. $p_i(x) \leq -1/2$).
%Therefore, for all $t=1,2,\dots,T$, if $y_t = i$ then $p_i(x_t) \ge 1/2$,
% and if $y_t \neq i$ then $p_i(x_t) \le -1/2$, and
%\begin{align*}
%&\norm{p_i}
%\\
%& \le
%\left[
%  188 \lceil \log_2(2K-2) \rceil \cdot \left \lceil \sqrt{{2}/{\gamma}} \right \rceil
%\right]^{
%  \frac{\lceil \log_2(2K-2) \rceil \cdot \left \lceil \sqrt{{2}/{\gamma}} \right \rceil }{2}
%} \; . \\
%\end{align*}
%By \autoref{lemma:norm-bound}, there exists $c_i \in \ell_2$ such that
%$\ip{c_i}{\phi(x)}_{\ell_2} = p_i(x)$, and
%\begin{align*}
%&\norm{c_i}_{\ell_2}
%\\
%& \le
%\left[
%  376 \lceil \log_2(2K-2) \rceil \cdot \left \lceil \sqrt{{2}/{\gamma}} \right \rceil
%\right]^{
%  \frac{\lceil \log_2(2K-2) \rceil \cdot \left \lceil \sqrt{{2}/{\gamma}} \right \rceil}{2}
%} \; .
%\end{align*}
%Define vectors $u_i \in \ell_2$ as
%\begin{align*}
%u_i & =
%\frac{c_i/\sqrt{K}}{
%  \left[
%    376 \lceil \log_2(2K-2) \rceil \cdot \left \lceil \sqrt{{2}/{\gamma}} \right \rceil
%  \right]^{
%    \frac{\lceil \log_2(2K-2) \rceil \cdot \left \lceil \sqrt{\frac{2}{\gamma}} \right \rceil}{2}
%  }
%} \; . \\
%\end{align*}

%Then, $\norm{u_1}^2 + \norm{u_2}^2 + \dots + \norm{u_K}^2 \le 1$.
%Futhermore, for all $t=1,2,\dots,T$, $\ip{u_{y_t}}{x_t} \ge \gamma_1$
%and for all $j \in \{1,2,\dots,K\} \setminus \{y_t\}$,
%$\ip{u_j}{x_t} \le - \gamma_1$. In other words,
%$(\phi(x_1), y_1), (\phi(x_2), y_2), \dots, (\phi(x_T), y_T)$ are
%strongly linearly separable with margin $\gamma_1 = \max\{\gamma_1, \gamma_2\}$.

%\paragraph{Case 2: $\gamma_1 \leq \gamma_2$.} Following the
%same idea as Case 1 and using Theorem~\ref{theorem:polynomial-approximation-1},
%we can show that
%$(\phi(x_1), y_1), (\phi(x_2), y_2), \dots, (\phi(x_T), y_T)$ are
%strongly linearly separable with margin $\gamma_2 = \max\{\gamma_1, \gamma_2\}$.

%In summary, the examples are strongly
%linearly separable with margin $\gamma' = \max\{\gamma_1, \gamma_2\}$.
%Finally, observe that for any $t=1,2,\dots,T$,
%\[
%k(x_t,x_t) = \frac{1}{1 - \frac{1}{2} \norm{x_t}^2} \le 2 \; .
%\qedhere
%\]
%\end{proof}
