\section{Nearest neighbor algorithm}
\begin{algorithm}
\caption{A nearest-neighbor algorithm for \textsc{Online Multiclass Linear Classification with
Bandit Feedback}}
\begin{algorithmic}[1]
\STATE $S \gets \emptyset$.
\FOR{$t=1,2,\ldots,T$:}
\IF{$\min_{(x,y) \in S} \| x_t - x \|_2 \leq \gamma$}
	\STATE $\hat{y}_t \gets \tilde{y}$, where $(\tilde{x}, \tilde{y}) = \argmin_{(x,y) \in S} \| x_t - x \|_2$.
\ELSE
	\STATE Predict $\hat{y}_t$ uniformly at random from $\cbr{1,\ldots,K}$, receive feedback $z_t = \one[\hat{y}_t \neq y_t]$.
  \label{line:nn-explore}
	\IF{$z_t = 0$}
    \STATE $S \gets S \cup \cbr{(x_t, \hat{y}_t)}$.
  \ENDIF
\ENDIF
\ENDFOR
\end{algorithmic}
\label{alg:nn}
\end{algorithm}

We present Algorithm~\ref{alg:nn} in this section. The algorithm is based on the following insight.

\begin{lemma}
Suppose $(x_1,y_1), \ldots, (x_T, y_T)$ is a set of $\gamma$-weakly linearly separable examples.
For $i$, $j$ in $\cbr{1,\ldots,T}$, if $x_i$ and $x_j$ have $\ell_2$ distance at most $\gamma$, then $y_i = y_j$.
\label{lem:nn}
\end{lemma}
\begin{proof}
Suppose for the sake on contradiction that $y_i \neq y_j$. By Definition~\ref{definition:weak-linear-separability}, there exists a set of $k$ vectors $(w_1, \ldots, w_k)$, such that Equations~\eqref{equation:weak-linear-separability-1}
and~\eqref{equation:weak-linear-separability-1} are satisfied.

Specifically,
\[ \ip{w_{y_i} - w_{y_j}}{x_i} \geq \gamma \]
\[ \ip{w_{y_j} - w_{y_i}}{x_j} \geq \gamma \]

This implies that
\[ \ip{w_{y_i} - w_{y_j}}{x_i - x_j} \geq 2\gamma \]

On the other hand,
\[ \ip{w_{y_i} - w_{y_j}}{x_i - x_j} \leq \| w_{y_i} - w_{y_j} \| \| x_i - x_j \| \leq \sqrt{2} \gamma \]
where the first inequality is from Cauchy-Schwarz,
the second inequality is from that
$\|w_{y_i} - w_{y_j}\| \leq \sqrt{2(\|w_{y_i}\|^2 + \|w_{y_j}\|^2)} \leq \sqrt{2}$ and our assumption
on $x_i$ and $x_j$. Therefore, we reach a contradiction.
\end{proof}


\begin{theorem}
Suppose $(x_1,y_1), \ldots, (x_T, y_T)$ is a set of $\gamma$-weakly linearly separable examples. In addition, for all $t$ in $\cbr{1,\ldots,T}$, $\| x_t \| \leq R$. Then, the expected number of mistakes made by Algorithm~\ref{alg:nn} is such that
\[\Exp[M] \leq (K-1) (\frac R \gamma)^d. \]
\label{thm:margin_at_upper}
\end{theorem}
\begin{proof}
Let $b_t$ be the indicator that line~\ref{line:nn-explore} is executed at timestep $t$,
i.e. we fall into the ``else'' case.
Note that if $b_t = 0$, then by Lemma~\ref{lem:nn}, the prediction $\hat{y}_t$ must equal $y_t$, i.e. $z_t = 0$.
Therefore, $M = \sum_{t=1}^T z_t = \sum_{t=1}^T b_t z_t$. Let $U = \sum_{t=1}^T b_t (1-z_t)$; it can be seen that $U$ is the number of times when $S$ gets updated. As $S$ is a $\gamma$-packing over
$\cbr{x: \|x\|_2 \leq R}$, the size of $S$ is at most $(\frac R \gamma)^d$, which is also an upper bound on $U$.

Note that when $b_t = 1$, $\hat{y}_t$ is chosen uniformly at random, we have
\[ \Exp[ z_t | b_t = 1] = \frac{K-1}{K} \]
Therefore,
\[ \Exp[M] = \Exp[\sum_{t=1}^T b_t z_t] = \frac{K-1}{K} \Exp[\sum_{t=1}^T b_t]. \]
On the other hand,
\[ \Exp[U] = \Exp[\sum_{t=1}^T b_t (1-z_t)] = \frac 1 K \Exp[\sum_{t=1}^T b_t]. \]
Therefore,
\[ \Exp[M] = (K-1) \Exp[U] \leq (K-1) (\frac R \gamma)^d. \]
\end{proof}
