\section{Proofs of Theorems~\ref{theorem:strongly-separable-examples-mistake-upper-bound} and \ref{theorem:strongly-separable-examples-mistake-lower-bound}}
\label{section:proofs-for-stringly-separable-examples}

%$\norm{x_1}, \norm{x_2}, \dots, \norm{x_T} \le R$
%$A
%= \sum_{t ~:~ S_t \neq \emptyset} z_t$
%B = \sum_{t ~:~ S_t = \emptyset} z_t
%\sum_{t ~:~ S_t = \emptyset}
\begin{proof}[Proof of \autoref{theorem:strongly-separable-examples-mistake-upper-bound}]
Let $M = \sum_{t=1}^T z_t$ be the number of
mistakes Algorithm~\ref{algorithm:algorithm-for-strongly-linearly-separable-examples} makes.
Let $A = \sum_{t=1}^T \one[S_t \neq \emptyset] z_t$ be the number of mistakes in the rounds
when $S_t \neq \emptyset$, i.e. the number of rounds line~\ref{line:neg-update} is
executed.
In addition, let $B = \sum_{t=1}^T \one[S_t = \emptyset] z_t$ be the
number of mistakes in the rounds when $S_t = \emptyset$.
It can be easily seen that $M = A + B$.

Let $C = \sum_{t=1}^T \one[S_t = \emptyset](1 - z_t)$ be the number of rounds
line~\ref{line:pos-update} gets executed. Let
$U = \sum_{t=1}^T (\one[S_t \neq \emptyset] z_t + \one[S_t = \emptyset](1 - z_t))$
be the number of rounds line~\ref{line:pos-update} or~\ref{line:neg-update}
gets executed. In other words, $U$ is the number of times the $K$-tuple of
vectors $(w_1^{(t)}, w_2^{(t)}, \dots, w_K^{(t)})$ gets updated. It can be
easily seen that $U = A + C$.

The key observation is that $\Exp[B] = (K-1) \Exp[C]$.
To see this, note that if $S_t = \emptyset$, there is $1/K$ probability that the algorithm
guesses the correct label ($z_t = 0$) and with probability $(K-1)/K$ algorithm's guess is
incorrect ($z_t = 1$). Therefore,
\[ \Exp[ z_t |S_t = \emptyset] = \frac{K-1}{K}, \]
\[ \Exp[B] = \frac{K-1}{K} \Exp \sbr{ \sum_{t=1}^T \indicator{S_t = \emptyset} }, \]
\[ \Exp[C] = \frac{1}{K} \Exp \sbr{ \sum_{t=1}^T \indicator{S_t = \emptyset} }. \]

Putting all the information together, we get that
\begin{align}
\Exp[M]
& = \Exp[A] + \Exp[B] \nonumber \\
& = \Exp[A] + (K-1) \Exp[C] \nonumber \\
& \le (K-1) \Exp[A + C] \nonumber\\
& = (K-1) \Exp[U]  \; .
\label{eqn:mistake-update}
\end{align}

To finish the proof, we need to upper bound the number of updates $U$. We claim
that $U \le \lfloor 4(R/\gamma)^2 \rfloor$ with probability 1.
The proof of this upper bound is
similar to the proof of the mistake bound for \textsc{Multiclass Perceptron}
algorithm. Let $w_1^*, w_2^*, \dots, w_K^* \in V$ be vectors that satisfy
\eqref{equation:strong-linear-separability-1},
\eqref{equation:strong-linear-separability-2} and
\eqref{equation:strong-linear-separability-3}.
The $K$-tuple $(w_1^{(t)}, w_2^{(t)}, \dots, w_K^{(t)})$
changes only if there is an update in round $t$.
We investigate how $\sum_{i=1}^K \norm{w_i^{(t)}}^2$ and
$\sum_{i=1}^K \ip{w_i^*}{w_i^{(t)}}$ change. If there is an update in round $t$,
by lines~\ref{line:pos-update} and~\ref{line:neg-update}, we always have
$ w_{\widehat y_t}^{(t+1)} = w_{\widehat y_t}^{(t)} + (-1)^{z_t} x_t $,
and for all $i \neq \widehat y_t$, $w_{i}^{(t+1)} = w_{i}^{(t)}$.
Therefore,
\begingroup
\allowdisplaybreaks
\begin{align*}
& \sum_{i=1}^K \norm{w_i^{(t+1)}}^2 \\
& = \left( \sum_{i \in \{1,2,\dots,K\} \setminus \{\widehat y_t\}} \norm{w_i^{(t)}}^2 \right) + \norm{w_{\widehat y_t}^{(t+1)}}^2 \\
& = \left( \sum_{i \in \{1,2,\dots,K\} \setminus \{\widehat y_t\}} \norm{w_i^{(t)}}^2 \right) + \norm{w_{\widehat y_t}^{(t)} + (-1)^{z_t} x_t}^2 \\
& = \left( \sum_{i=1}^K \norm{w_i^{(t)}}^2 \right) + \norm{x_t}^2 + \underbrace{(-1)^{z_t} 2 \ip{w_{\widehat y_t}^{(t)}}{x_t}}_{\le 0} \\
& \le \left( \sum_{i=1}^K \norm{w_i^{(t)}}^2 \right) + \norm{x_t}^2 \\
& \le \left( \sum_{i=1}^K \norm{w_i^{(t)}}^2 \right) + R^2 \; .
\end{align*}
\endgroup
The inequality that $(-1)^{z_t} 2 \ip{w_{\widehat y_t}^{(t)}}{x_t} \leq 0$ is from a case analysis: if line~\ref{line:pos-update} is executed, then $z_t = 0$ and $\ip{w_i^{(t)}}{x_t} \ge 0$; otherwise line~\ref{line:neg-update} is executed, in which case $z_t = 1$ and $\ip{w_i^{(t)}}{x_t} \le 0$.


Hence, after $U$ updates,
\begin{equation}
\sum_{i=1}^K \norm{w_i^{(T+1)}}^2 \le R^2 U \; .
\label{eqn:norm-ub}
\end{equation}
Similarly, if there is an update in round $t$, we have
\begingroup
\allowdisplaybreaks
\begin{align*}
& \sum_{i=1}^K \ip{w_i^*}{w_i^{(t)}} \\
& = \left( \sum_{i \in \{1,2,\dots,K\} \setminus \{\widehat y_t\}} \ip{w_i^*}{w_i^{(t)}} \right) + \ip{w_{\widehat y_t}^*}{w_{\widehat y_t}^{(t+1)}} \\
& = \left( \sum_{i \in \{1,2,\dots,K\} \setminus \{\widehat y_t\}} \ip{w_i^*}{w_i^{(t)}} \right) \\
& \qquad + \ip{w_{\widehat y_t}^*}{w_{\widehat y_t}^{(t)} + (-1)^{z_t} x_t} \\
& = \left( \sum_{i=1}^K \ip{w_i^*}{w_i^{(t)}} \right) + (-1)^{z_t} \ip{w_{\widehat y_t}^*}{x_t} \\
& \ge \left( \sum_{i=1}^K \ip{w_i^*}{w_i^{(t)}} \right) + \frac \gamma 2,
\end{align*}
\endgroup
where the last inequality follows from a case analysis on $z_t$ and
Definition~\ref{definition:strong-linear-separability}: if $z_t = 0$, then
$\widehat y_t = y_t$, by Equation~\eqref{equation:strong-linear-separability-2},
we have that $\ip{w_{\widehat y_t}^*}{x_t} \geq \frac \gamma 2$; if $z_t = 1$,
then $\widehat y_t \neq y_t$, by
Equation~\eqref{equation:strong-linear-separability-3}, we have that
$\ip{w_{\widehat y_t}^*}{x_t} \le -\frac \gamma 2$.

Thus, after $U$ updates,
\begin{equation}
\sum_{i=1}^K \ip{w_i^*}{w_i^{(T+1)}} \ge \frac {\gamma U} 2 \; .
\label{eqn:norm-lb}
\end{equation}
Applying Cauchy-Schwartz's inequality twice, and using assumption
\eqref{equation:strong-linear-separability-1}, we get that
\begin{align*}
\sum_{i=1}^K \ip{w_i^*}{w_i^{(T+1)}}
& \le \sum_{i=1}^K \norm{w_i^*} \cdot \norm{w_i^{(T+1)}} \\
& \le \sqrt{\sum_{i=1}^K \norm{w_i^*}^2} \sqrt{\sum_{i=1}^K \norm{w_i^{(T+1)}}^2} \\
& \le \sqrt{\sum_{i=1}^K \norm{w_i^{(T+1)}}^2} \; .
\end{align*}
Combining the above inequality with Equations~\eqref{eqn:norm-ub} and~\eqref{eqn:norm-lb}, we get
$$
\left(\frac{\gamma U}{2} \right)^2 \le \sum_{i=1}^K \norm{w_i^{(T+1)}}^2 \le R^2 U \; .
$$
We conclude that $U \le 4(R/\gamma)^2$. Since $U$ is an integer, $U \le \lfloor 4(R/\gamma)^2 \rfloor$.

Applying Equation~\eqref{eqn:mistake-update}, we get
$$
\Exp[M] \leq (K-1) \Exp[U] \leq (K-1) \lfloor 4(R/\gamma)^2 \rfloor \; . \qedhere
$$
\end{proof}




\begin{proof}[Proof of \autoref{theorem:strongly-separable-examples-mistake-lower-bound}]
We use probabilistic method. Let $M = \left\lfloor \frac{1}{4} (R/\gamma)^2
\right\rfloor$. Let $V = \R^{M+1}$ equipped with the standard inner product.
Let $e_1, e_2, \dots, e_{M+1}$ be the standard orthonormal basis of $V$. We
define vectors $v_1, v_2, \dots, v_M \in V$ where $v_j = \frac{R}{\sqrt{2}}(e_j
+ e_{M+1})$ for $j=1,2,\dots,M$. Let $\ell_1, \ell_2, \dots, \ell_M$ be chosen
i.i.d. uniformly at random from $\{1,2,\dots,K\}$ and independently of any
randomness used the by algorithm $\calA$. Let $T = M (K - 1)$. We define examples $(x_1,
y_1), (x_2, y_2), \dots, (x_T, y_T)$ as follows. For any $j=1,2,\dots,M$ and any
$h=1,2,\dots,K-1$,
$$
(x_{(j-1)(K-1) + h}, y_{(j-1)(K-1) + h}) = (v_j, \ell_j)
$$


With probability one, the norm of each example is exactly $R$.
We show that with probability
one, the examples are strongly separable with margin $\gamma$. To see
that, consider $w_1^*, w_2^*, \dots, w_K^* \in V$ defined by
$$
w_i^* = \sqrt{2} \frac{\gamma}{R} \left( \sum_{j ~:~ \ell_j = i} e_j \right) - \frac{\sqrt{2}}2 \frac{\gamma}{R} e_{M+1}
$$
for $i=1,2,\dots,K$.

for $i \in \{1,2,\dots,K\}$ and $j \in
\{1,2,\dots,M\}$, we consider the inner product of $w_i^*$ and $v_j$.
If $i = \ell_j$, $\ip{w_i^*}{v_j} = \gamma - \frac \gamma 2 = \frac \gamma 2$;
otherwise $i \neq \ell_j$, in which case
$\ip{w_i^*}{v_j} = 0 - \frac \gamma 2 = - \frac \gamma 2$.
This means that $w_1^*, w_2^*, \dots, w_K^*$ satisfy
conditions
\eqref{equation:strong-linear-separability-2} and
\eqref{equation:strong-linear-separability-3}. Condition \eqref{equation:strong-linear-separability-1}
is satisfied since
\begin{multline*}
\sum_{i=1}^K \norm{w_i^*}^2
= 2 \frac{\gamma^2}{R^2} \sum_{j=1}^M \norm{e_j}^2 +  \frac{\gamma^2}{2 R^2} K \norm{e_{M+1}}^2 \\
= 2 \frac{\gamma^2}{R^2} M + \frac{\gamma^2}{2 R^2} K
\le \frac{1}{2} + \frac{1}{2}
= 1 \; .
\end{multline*}

It remains to lower bound the expected number of mistakes of $\calA$. For
any $j \in \{1,2,\dots,M\}$, consider the expected number of mistakes the
algorithm makes in rounds $(K-1)(j-1) + 1, (K-1)(j-1) + 2, \dots, (K-1)j$.

Define a filtration $\cbr{\calB_j}_{j=0}^M$,
where $\calB_j = \sigma((x_1, y_1, \hat{y}_1), \ldots, (x_{(K-1)j}, y_{(K-1)j}, \hat{y}_{(K-1)j}))$
for every $j$ in $\{1,2,\dots,M\}$.
By Claim 2 of~\citet{Daniely-Helbertal-2013}, as $\ell_j$
is chosen uniformly from $\cbr{1,\dots,K}$ and independent of $\calB_{j-1}$ and $\calA$'s
randomness,
$$
\Exp \sbr{ \sum_{t=(K-1)(j-1) + 1}^{(K-1)j} z_t ~\Bigg|~ \calB_{j-1} } \geq \frac{K-1}2 \; .
$$
This implies that
$$
\Exp \sbr{ \sum_{t=(K-1)(j-1) + 1}^{(K-1)j} z_t } \ge \frac{K-1}2 \; .
$$
Summing over all $j$ in $\{1,2,\dots,M\}$,
$$
\Exp \sbr{ \sum_{t=1}^{(K-1)M} z_t} \geq \frac{K-1}2 \cdot M = \frac{K-1}2 \left\lfloor \frac 1 4 (R/\gamma)^2 \right\rfloor \; .
$$

%In these rounds,
%the algorithm is guessing the label $\ell_j$. Since $\ell_j$ is chosen uniformly
%at random from the set $\{1,2,\dots,K\}$ and feedback is only binary, the
%expected number of mistakes the algorithm makes in these rounds is at least
%$\frac{K-1}{2}$. Altogether, the algorithm makes at least $\frac{K-1}{2} M$
%mistakes in expectation.

%\chicheng{I think this part need to be formalized, by directly using
%~\cite{Daniely-Helbertal-2013}, Claim 2.}
%Can we assume without loss of generality
%that a deterministic prediction algorithm will have a smaller expected cost?
%If so, define $F(k)$ be the number of mistake that will be made by the algorithm,
%if the adversary commits to a label uniformly at random from $\cbr{1,2,\ldots,k}$,
%and the guessing of the label lasts for $k$ rounds.
%we can construct a recurrence of $F$: $F(k) = F(k-1) + \frac{k-1}{k}$, and $F(1) = 0$.
%Therefore, $F(K) = \Omega(K)$.

Finally, since strong separability and norm condition hold with probability one,
there exists a particular (i.e. deterministic) sequence of examples for which
the algorithm makes at least $\frac{K-1}2 \left\lfloor \frac 1 4 (R/\gamma)^2
\right\rfloor$ mistakes in expectation over its internal randomization.
\end{proof}
