\section{Related work}
\label{section:related-work}

The problem of bandit multiclass linear classification was initially formulated
by the pioneering work of~\citet{Kakade-Shalev-Shwartz-Tewari-2008}. In that
work, two computationally inefficient algorithms that work in the standard
linearly separable setting are proposed: one with a mistake bound of $O(K^2 d
\ln(\frac{d}{\gamma^2}))$, the other with a mistake bound of
$\widetilde{O}(\frac{K^2}{\gamma^2} \ln T)$. In addition, the authors propose
the \textsc{Banditron} algorithm, a computationally efficient algorithm that has
a $O(T^{2/3})$ regret against the multiclass hinge loss in the general setting,
and has a $O(\sqrt{T})$ mistake bound in the $\gamma$-weak linearly separable
setting. The polynomial dependence of \textsc{Banditron}'s mistake bound on the
time horizon is undesirable for problems with a long time horizon. One key open
question left by~\citet{Kakade-Shalev-Shwartz-Tewari-2008} is whether one can
design computationally efficient algorithms that achieve mistake bounds that
match or improve over those of inefficient algorithms. In this paper, we take a
step towards answering this question, showing that efficient algorithms with
mistake bounds quasipolynomial in the margin parameter can be obtained.

%Whether one can design an efficient algorithm with a finite mistake bound that
%has no dimensionality dependence is mentioned as an open problem in
%~\cite{Kakade-Shalev-Shwartz-Tewari-2008}, where in this paper we provide a
%positive answer. $O(k^2 d \ln(\frac{d}{\gamma^2}))$

Many works consider bandit multiclass classification in the general
non-separable setting. \citet{Abernethy-Rakhlin-2009} poses the open problem of
whether one can design an efficient algorithm to get a $O(\sqrt{T})$ regret
against some reasonable loss functions. \citet{Crammer-Gentile-2013} show that
such an algorithm can be obtained, provided that the distribution of the label
$y_t$ conditioned on feature vector $x_t$ satisfies certain parametric noise
assumption. \citet{Hazan-Kale-2011} developed the \textsc{Newtron} algorithm,
which has a regret between $O(\ln T)$ and $O(T^{2/3})$ against the multiclass
logistic loss, where the exact order of the regret depends on the diameter of
the benchmark class. In particular, if the diameter is $O(\ln T)$, its regret
bound would become $O(T^{2/3})$. The \textsc{SOBA} algorithm by
\citet{Beygelzimer-Orabona-Zhang-2017} achieves a regret of
$\widetilde{O}(\sqrt{T})$ against the
$\eta$-loss~\citet{Orabona-Cesa-Bianchi-Gentile-2012}. In addition, its regret
bound does not depend sensitively on the diameter of the benchmark class.
\citet{Foster-Kale-Luo-Mohri-Sridharan-2018} developed an algorithm that has a
regret of $\widetilde{O}(\sqrt{T})$ against the multiclass logistic loss, where
it doubly-exponentially improves over \citet{Hazan-Kale-2011}'s regret on its
dependence on the diameter of the benchmark class.

\citet{Chen-Lin-Lu-2014} and \citet{Zhang-Jung-Tewari-2018} study online bandit
multiclass boosting under bandit feedback, where one can view online boosting as
online linear classification by treating each base hypothesis as a separate
feature. \citet{Chen-Lin-Lu-2014}'s online weak learning condition implies that
the set of examples is strongly separable by a convex combination of base
hypotheses with a margin. Under this condition, it gives an algorithm with a
$O(T^{3/4})$ mistake bound. \citet{Zhang-Jung-Tewari-2018} considers an online
weak learning condition, which implies that all examples is separable by a
convex combination of base hypotheses with a margin
(See~\cite{Mukherjee-Schapire-2013}, Theorem 3). Under this condition, it gives
a boosting algorithm with a $O(\sqrt{T})$ mistake bound with the knowledge of
the edge parameter. In addition, it gives an adaptive algorithm with a
$O(T^{3/4})$ mistake bound.

The online bandit multiclass learning is a special case of the contextual
bandits problem~\citep{Auer-2003, Langford-Zhang-2008}, where the cost is the
classification error. A series of works in the contextual bandits learning
literature focuses on oracle-efficiency, that is, they assume access to an
oracle that given a set of cost-sensitve learning examples.

There has been much literature on the contextual bandit learning
problem~\citep{Auer-2003, Langford-Zhang-2008}. In this problem, the learner is
given a policy class $\Pi$, and at each timestep $t$, an example $x_t$ is shown,
with an associated cost vector $c_t$ hidden. The learner then takes an action
$\hat{y}_t$, and observed the incurred cost $c_t$. The goal of the learner is to
minimize its regret $\sum_{t=1}^T c_t(\hat{y}_t) - \min_{\pi \in \Pi}
\sum_{t=1}^T c_t(\pi(x_t))$. The online bandit multiclass learning problem is a
special case of the contextual bandits problem, where the cost $c_t(i)$ is the
classification error $I(i \neq y_t)$, and the policy class $\Pi$ is the set of
linear classifiers $\cbr{x \mapsto \argmax_y (Wx)_y: W \in \R^{k \times d}}$. A
series of works in the contextual bandits learning literature focus on
oracle-efficiency~\citep{Dudik-Hsu-Kale-Karampatziakis-Langford-Reyzin-Zhang-2011,
Agarwal-Hsu-Kale-Langford-Li-Schapire-2014, Rakhlin-Sridharan-2016,
Syrgkanis-Krishnamurthy-Schapire-2016,
Syrgkanis-Luo-Krishnamurthy-Schapire-2016}. Specifically, they assume access to
a policy optimization oracle that receives a set of cost-sensitive learning
examples and policy class $\Pi$, return the policy in $\Pi$ that minimizes its
empirical cost on the examples, and the algorithm only call the oracle a
polynomial number of times. However, these algorithms are not truly
computationally efficient in our setting, as it is NP-hard in general to find a
linear classifier that minimizes the empirical cost over a set of
examples~\citep{Arora-Babai-Stern-Sweedyk-1997}.

%as algorithms generate cost-sensitive examples that may not be linearly
%separable Algorithms that satisfy oracle efficiency has been proposed in the
%literature, for example.

Recently, \citet{Foster-Krishnamurthy-2018} developed a rich theory of
contextual bandits with surrogate losses, focusing on benchmarks of the form
$\min_{f \in \calF} \sum_{t=1}^T \frac{1}{K} \sum_{i=1}^K c_t(i) \phi( f_i(x_t)
)$, where $\calF$ contains score functions $f$ such that $\sum_{i=1}^K
f_i(\cdot) \equiv 0$, and $\phi(s) = \max(1 - \frac s \gamma, 0)$ or $\min(1,
\max(1 - \frac s \gamma, 0))$. On one hand, it gives information-theoretic
mistake upper bounds under various settings of $\calF$, including parametric and
nonparametric classes. On the other hand, it gives an efficient algorithm that
has a $O(\sqrt{T})$ regret against the benchmark of $\calF = \cbr{x \mapsto W x:
W \in \R^{k \times d}, \one^T W = 0}$. A direct application of this result to
the online bandit multiclass learning problem yields an algorithm with a
$O(\sqrt{T})$ mistake bound in the strong linearly separable setting.

%\cite{Chen-Chen-Zhang-Chen-Zhang-2009} studies the approach of reducing bandit
%multiclass learning to online binary classification using one-versus-all
%reduction. They show that

%Their results do not imply a finite mistake bound in the weakly separable
%setting, in that the benchmark loss can still be $\Omega(T)$.
