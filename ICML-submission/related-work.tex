\section{Related work}
\label{section:related-work}

The problem of online bandit multiclass learning was initially formulated in the
pioneering work of~\citet{Auer-Long-1999} under the name of ``weak reinforcement
model''. They showed that if all examples agree with some classifier from a
prespecified hypothesis class $\calH$, then the optimal mistake bound in the
bandit setting can be upper bounded by the optimal mistake bound in the full
information setting, times a factor of $(2.01 + o(1))K \ln K$. \citet{Long-2017}
later improved the factor to $(1 + o(1)) K \ln K$ and showed its
near-optimality. \citet{Daniely-Helbertal-2013} extended the results to the
setting where the performance of the algorithm is measured by its regret, i.e.
the difference between the number of mistakes made by the algorithm and the
number of mistakes made by the best classifier in $\calH$ in hindsight. We
remark that all algorithms developed in this context are computationally
inefficient.

The linear classification version of this problem is initially studied
by~\citet{Kakade-Shalev-Shwartz-Tewari-2008}. They proposed two computationally
inefficient algorithms that work in the weakly linearly separable setting, one
with a mistake bound of $O(K^2 d \ln(d/\gamma))$, the other with a mistake bound
of $\widetilde{O}((K^2/\gamma^2) \ln T)$. The latter result was later improved
by \citet{Daniely-Helbertal-2013}, which gives a computationally inefficient
algorithm with a mistake upper bound of $\widetilde{O}(K/\gamma^2)$. In
addition,~\citet{Kakade-Shalev-Shwartz-Tewari-2008} propose the
\textsc{Banditron} algorithm, a computationally efficient algorithm that has a
$O(T^{2/3})$ regret against the multiclass hinge loss in the general setting,
and has a $O(\sqrt{T})$ mistake bound in the $\gamma$-weakly linearly separable
setting. In contrast to mild dependencies on the time horizon for mistake bounds
of computationally inefficient algorithms, the polynomial dependence of
\textsc{Banditron}'s mistake bound on the time horizon is undesirable for
problems with a long time horizon, in the weakly linearly separable setting. One
key open question left by~\citet{Kakade-Shalev-Shwartz-Tewari-2008} is whether
one can design computationally efficient algorithms that achieve mistake bounds
that match or improve over those of inefficient algorithms. In this paper, we
take a step towards answering this question, showing that efficient algorithms
with mistake bounds quasipolynomial in $1/\gamma$ (for constant $K$) or
quasipolynomial in $K$ (for constant $\gamma$) can be obtained.

%In addition, it also
% shows that under different relationships between $k$, $d$ and $\gamma$,
% the adversary can force the learner to incur $\Omega(K^2 d)$
% or $\Omega(\frac{K}{\gamma^2})$ mistakes.

%Whether one can design an efficient algorithm with a finite mistake bound that
%has no dimensionality dependence is mentioned as an open problem in
%~\cite{Kakade-Shalev-Shwartz-Tewari-2008}, where in this paper we provide a
%positive answer. $O(k^2 d \ln(\frac{d}{\gamma^2}))$

The general problem of linear bandit multiclass learning has received considerable attention~\cite{Abernethy-Rakhlin-2009, Wang-Jin-Valizadegan-2010,
Crammer-Gentile-2013, Hazan-Kale-2011, Beygelzimer-Orabona-Zhang-2017,
Foster-Kale-Luo-Mohri-Sridharan-2018}. \citet{Chen-Lin-Lu-2014,
Zhang-Jung-Tewari-2018} study online bandit multiclass boosting under bandit
feedback, where one can view boosting as linear classification by treating each
base hypothesis as a separate feature.  
%However, most of these works achieve a
%regret of order $\widetilde{O}(\sqrt{T})$ to $O(T^{3/4})$.  
In the weakly linearly separable setting, however, 
all these algorithms guarantee a mistake 
bound of ${O}(\sqrt{T})$ at best.
%One exception is the \textsc{Newtron} algorithm, which---with
%an appropriate setting of parameter $\alpha$---can achieve a $O(\ln T)$
%regret against the $\alpha$-logistic loss. However, as argued in
%~\citep[][Appendix D]{Beygelzimer-Orabona-Zhang-2017}, if the setting of
%$\alpha$ makes the regret bound $O(\sqrt{T})$, the $\alpha$-logistic loss of the
%best linear predictor would be $\Omega(T^{3/4})$. Therefore, \textsc{Newtron} at
%best guarantees a mistake upper bound of $O(\sqrt{T})$ in the weakly linearly
%separable setting.

%\citet{Abernethy-Rakhlin-2009} poses the open problem of whether one can design
%an efficient algorithm to get a $O(\sqrt{T})$ regret against some reasonable
%loss functions. \citet{Crammer-Gentile-2013} show that such an algorithm can be
%obtained, provided that the distribution of the label $y_t$ conditioned on
%feature vector $x_t$ satisfies certain parametric noise assumption.
%\citet{Hazan-Kale-2011} developed the \textsc{Newtron} algorithm, which has a
%regret between $O(\ln T)$ and $O(T^{2/3})$ against the multiclass logistic
%loss, where the exact order of the regret depends on the diameter of the
%benchmark class. In particular, if the diameter is $O(\ln T)$, its regret bound
%would become $O(T^{2/3})$. The \textsc{SOBA} algorithm by
%\citet{Beygelzimer-Orabona-Zhang-2017} achieves a regret of
%$\widetilde{O}(\sqrt{T})$ against the
%$\eta$-loss~\citet{Orabona-Cesa-Bianchi-Gentile-2012}. In addition, its regret
%bound does not depend sensitively on the diameter of the benchmark class.
%\citet{Foster-Kale-Luo-Mohri-Sridharan-2018} developed an algorithm that has a
%regret of $\widetilde{O}(\sqrt{T})$ against the multiclass logistic loss, where
%it doubly-exponentially improves over \citet{Hazan-Kale-2011}'s regret on its
%dependence on the diameter of the benchmark class.

%\citet{Chen-Lin-Lu-2014}'s online weak learning condition implies that the set
%of examples is strongly separable by a convex combination of base hypotheses
%with a margin. Under this condition, it gives an algorithm with a $O(T^{3/4})$
%mistake bound. \citet{Zhang-Jung-Tewari-2018} considers an online weak learning
%condition, which implies that all examples is separable by a convex combination
%of base hypotheses with a margin; see~\citet[Theorem
%3]{Mukherjee-Schapire-2013}. Under this condition, it gives a boosting
%algorithm with a $O(\sqrt{T})$ mistake bound with the knowledge of the edge
%parameter. In addition, it gives an adaptive algorithm with a $O(T^{3/4})$
%mistake bound.

There is a large body of literature on contextual bandit
learning~\citep{Auer-2003, Langford-Zhang-2008}. In this problem, the learner is
given a policy class $\Pi$, and at each timestep $t$, an example $x_t$ is shown,
with an associated cost vector $c_t$ hidden. The learner then takes an action
$\widehat{y}_t$, and observes the incurred cost $c_t(\widehat{y}_t)$. The goal
of the learner is to minimize its regret $\sum_{t=1}^T c_t(\widehat{y}_t) -
\min_{\pi \in \Pi} \sum_{t=1}^T c_t(\pi(x_t))$. The bandit multiclass learning
problem can be seen as a special case, where the cost $c_t(i)$ is the
classification error $\indicator{i \neq y_t}$, and the policy class $\Pi$ is the
set of linear classifiers $\cbr{x \mapsto \argmax_y (Wx)_y: W \in \R^{k \times
d}}$. A series of papers in the contextual bandits learning literature assume
access to a policy optimization oracle that returns a policy in $\Pi$ that
minimizes its empirical cost on any set of cost-sensitive examples, and
construct algorithms that call the oracle a polynomial number of times%
~\citep{Dudik-Hsu-Kale-Karampatziakis-Langford-Reyzin-Zhang-2011,
Agarwal-Hsu-Kale-Langford-Li-Schapire-2014, Rakhlin-Sridharan-2016,
Syrgkanis-Krishnamurthy-Schapire-2016,
Syrgkanis-Luo-Krishnamurthy-Schapire-2016}. However, these algorithms are not
truly computationally efficient in our setting, as it is NP-hard in general to
find a linear classifier that minimizes the empirical cost over a set of
examples~\citep{Arora-Babai-Stern-Sweedyk-1997}.

Recently, \citet{Foster-Krishnamurthy-2018} developed a rich theory of
contextual bandits with surrogate losses, focusing on regrets of the form
$\sum_{t=1}^T c_t(\widehat{y}_t) - \min_{f \in \calF} \sum_{t=1}^T \frac{1}{K}
\sum_{i=1}^K c_t(i) \phi( f_i(x_t) )$, where $\calF$ contains score functions
$f$ such that $\sum_{i=1}^K f_i(\cdot) \equiv 0$, and $\phi(s) = \max(1 - \frac
s \gamma, 0)$ or $\min(1, \max(1 - \frac s \gamma, 0))$. On one hand, it gives
information-theoretic regret upper bounds for various settings of $\calF$. On
the other hand, it gives an efficient algorithm with an $O(\sqrt{T})$ regret
against the benchmark of $\calF = \cbr{x \mapsto W x: W \in \R^{k \times d},
\one^T W = 0}$. A direct application of this result to \textsc{Online Bandit
Multiclass Linear Classification} gives an algorithm with $O(\sqrt{T})$ mistake
bound in the strongly linearly separable case.

%including parametric and nonparametric classes

%\cite{Chen-Chen-Zhang-Chen-Zhang-2009} studies the approach of reducing bandit
%multiclass learning to online binary classification using one-versus-all
%reduction. They show that

%Their results do not imply a finite mistake bound in the weakly separable
%setting, in that the benchmark loss can still be $\Omega(T)$.
