\section{Full proof of Theorem~\ref{theorem:margin-transformation}}
\label{section:full-proof-of-theorem-margin-transformation}

%\begin{proof}[Proof of Theorem~\ref{theorem:margin-transformation}]
%First osberve that for any $t=1,2,\dots,T$,
%$$
%\norm{\phi(x_t)}_{\ell_2} = k(x_t,x_t) = \frac{1}{1 - \frac{1}{2} \norm{x_t}^2} \le 2 \; .
%$$

%Since the examples $(x_1, y_1), (x_2, y_2), \dots, (x_T, y_T)$ are weakly
%linearly separable with margin $\gamma$ there are vectors $w_1, w_2, \dots, w_K$
%satisfying \eqref{equation:weak-linear-separability-1} and
%\eqref{equation:weak-linear-separability-2}.

%Fix any $i \in \{1,2,\dots,K\}$. Consider the $K-1$ vectors $(w_i - w_j)/2$ for
%$j \in \{1,2,\dots,K\} \setminus \{i\}$. Note that the vectors have norm at most
%$1$.
%Theorems~\ref{theorem:polynomial-approximation-1}~and~\ref{theorem:polynomial-approximation-2}
%imply that there exist multivariate polynomials $p_i:\R^d \to \R$ and $q_i:\R^d
%\to \R$ such that
%\begin{align*}
%\deg(p_i) & = \lceil \log_2(2K-2) \rceil \cdot \left\lceil \sqrt{\frac{2}{\gamma}} \right\rceil \; , \\
%\deg(q_i) & = (2s+1) r(K-1) \; .
%\end{align*}
%Futhermore, for all $t=1,2,\dots,T$, if $y_t = i$ then $p_i(x_t) \ge 1/2$,
%$q_i(x_t) \ge \frac{1}{4} \cdot 2^{s(s+1)r(K-1)}$ and if $y_t \neq i$ then
%$p_i(x_t) \le -1/2$, $q_i(x) \le - \frac{1}{4} \cdot 2^{s(s+1)r(K-1)}$ and
%\begin{align*}
%\norm{p_i} & \le \left(188 \lceil \log_2(2K-2) \rceil \cdot \left \lceil \sqrt{\frac{2}{\gamma}} \right \rceil \right)^{\frac{1}{2} \lceil \log_2(2K-2) \rceil
%\cdot \left \lceil \sqrt{\frac{2}{\gamma}} \right \rceil} \; , \\
%\norm{q_i} & \le (2K-5/2) 2^{K-1} \\
%& \qquad \cdot \left(2^{2s} r(K-1) (4s+2)^2 \right)^{(s+1/2)r(K-1)} \; .
%\end{align*}
%By \autoref{lemma:norm-bound} there exists $c_i, c_i' \in \ell_2$ such that
%$\ip{c_i}{\phi(x)} = p_i(x)$ and $\ip{c_i'}{\phi(x)} = q_i(x)$ and
%\begin{align*}
%\norm{c_i}_{\ell_2}
%& \le \left(376 \lceil \log_2(2K-2) \rceil \cdot \left \lceil \sqrt{\frac{2}{\gamma}} \right \rceil \right)^{\frac{1}{2} \lceil \log_2(2K-2) \rceil
%\cdot \left \lceil \sqrt{\frac{2}{\gamma}} \right \rceil} \\
%\norm{c'_i}_{\ell_2} & \le (2K-5/2) 2^{K-1} \\
%& \qquad \cdot \left(2^{2s+1} r(K-1) (4s+2)^2 \right)^{(s+1/2)r(K-1)} \; .
%\end{align*}
%Define vectors $u_i, u_i' \in \ell_2$ as
%\begin{align*}
%u_i & = \frac{c_i}{\sqrt{K} \left(376 \lceil \log_2(2K-2) \rceil \cdot \left \lceil \sqrt{\frac{2}{\gamma}} \right \rceil \right)^{\frac{1}{2} \lceil \log_2(2K-2) \rceil
%\cdot \left \lceil \sqrt{\frac{2}{\gamma}} \right \rceil}} \; . \\
%u_i' & = \frac{c_i'  \cdot \left(2^{2s+1} r(K-1) (4s+2)^2 \right)^{-(s+1/2)r(K-1)}}{\sqrt{K} (2K-5/2) 2^{K-1}} \; .
%\end{align*}

%Then, $\norm{u_1}^2 + \norm{u_2}^2 + \dots + \norm{u_K}^2 \le 1$ and
%$\norm{u_1'}^2 + \norm{u_2'}^2 + \dots + \norm{u_K'}^2 \le 1$.
%Futhermore, for all $t=1,2,\dots,T$, $\ip{u_{y_t}}{x_t} \ge \gamma_1$ and
%$\ip{u'_{y_t}}{x_t} \ge \gamma_2$
%and for all $j \in \{1,2,\dots,K\} \setminus \{y_t\}$,
%$\ip{u_j}{x_t} \le - \gamma_1$ and $\ip{u'_j}{x_t} \le - \gamma_2$. In other words,
%$(\phi(x_1), y_1), (\phi(x_2), y_2), \dots, (\phi(x_T), y_T)$ are
%strongly linearly separable with margin $\gamma_1$ and also strongly linearly
%separable with margin $\gamma_2$. Therefore, the examples are strongly
%linearly separable with margin $\gamma' = \max\{\gamma_1, \gamma_2\}$.
%\end{proof}

\begin{proof}[Proof of Theorem~\ref{theorem:margin-transformation}]
Since the examples $(x_1, y_1)$, $(x_2, y_2)$, $\dots$, $(x_T, y_T)$ are weakly
linearly separable with margin $\gamma$,, there are vectors $w_1, w_2, \dots, w_K$
satisfying \eqref{equation:weak-linear-separability-1} and
\eqref{equation:weak-linear-separability-2}.

Fix any $i \in \{1,2,\dots,K\}$. Consider the $K-1$ vectors $(w_i - w_j)/2$ for
$j \in \{1,2,\dots,K\} \setminus \{i\}$. Note that the vectors have norm at most
$1$. We consider two cases regarding the relationship between $\gamma_1$ and
$\gamma_2$.

\paragraph{Case 1: $\gamma_1 \geq \gamma_2$.} In this setting, Theorem~\ref{theorem:polynomial-approximation-1}
implies that there exist a multivariate polynomial $p_i:\R^d \to \R$,
\begin{align*}
\deg(p_i) & = \lceil \log_2(2K-2) \rceil \cdot \left\lceil \sqrt{\frac{2}{\gamma}} \right\rceil \; ,
\end{align*}
such that all examples $x$ in $R_i^+$ (resp. $R_i^-$) satisfy $p_i(x) \geq 1/2$
(resp. $p_i(x) \leq -1/2$).
Therefore, for all $t=1,2,\dots,T$, if $y_t = i$ then $p_i(x_t) \ge 1/2$,
 and if $y_t \neq i$ then $p_i(x_t) \le -1/2$, and
\begin{align*}
\norm{p_i} & \le \left(188 \lceil \log_2(2K-2) \rceil \cdot \left \lceil \sqrt{\frac{2}{\gamma}} \right \rceil \right)^{\frac{1}{2} \lceil \log_2(2K-2) \rceil
\cdot \left \lceil \sqrt{\frac{2}{\gamma}} \right \rceil} \; , \\
\end{align*}
By \autoref{lemma:norm-bound}, there exists $c_i \in \ell_2$ such that
$\ip{c_i}{\phi(x)} = p_i(x)$, and
\begin{align*}
\norm{c_i}_{\ell_2}
& \le \left(376 \lceil \log_2(2K-2) \rceil \cdot \left \lceil \sqrt{\frac{2}{\gamma}} \right \rceil \right)^{\frac{1}{2} \lceil \log_2(2K-2) \rceil
\cdot \left \lceil \sqrt{\frac{2}{\gamma}} \right \rceil} \; .
\end{align*}
Define vectors $u_i \in \ell_2$ as
\begin{align*}
u_i & = \frac{c_i}{\sqrt{K} \left(376 \lceil \log_2(2K-2) \rceil \cdot \left \lceil \sqrt{\frac{2}{\gamma}} \right \rceil \right)^{\frac{1}{2} \lceil \log_2(2K-2) \rceil
\cdot \left \lceil \sqrt{\frac{2}{\gamma}} \right \rceil}} \; . \\
\end{align*}

Then, $\norm{u_1}^2 + \norm{u_2}^2 + \dots + \norm{u_K}^2 \le 1$.
Furthermore, for all $t=1,2,\dots,T$, $\ip{u_{y_t}}{x_t} \ge \gamma_1$
and for all $j \in \{1,2,\dots,K\} \setminus \{y_t\}$,
$\ip{u_j}{x_t} \le - \gamma_1$. In other words,
$(\phi(x_1), y_1), (\phi(x_2), y_2), \dots, (\phi(x_T), y_T)$ are
strongly linearly separable with margin $\gamma_1 = \max\{\gamma_1, \gamma_2\}$.

\paragraph{Case 2: $\gamma_1 < \gamma_2$.} In this setting, Theorem~\ref{theorem:polynomial-approximation-2}
implies that there exist a multivariate polynomial $q_i:\R^d \to \R$,
\begin{align*}
\deg(q_i) & = (2s+1) r(K-1) \; ,
\end{align*}
such that all examples $x$ in $R_i^+$ (resp. $R_i^-$) satisfy $q_i(x) \geq 1/2$
(resp. $q_i(x) \leq -1/2$). Recall that here,
\[
r = 2 \left\lceil \frac{1}{4} \log_2(4K - 3) \right\rceil + 1 \quad \text{and} \quad s = \left \lceil \log_2(1/\gamma) \right \rceil \; .
\]


Therefore, for all $t=1,2,\dots,T$, if $y_t = i$ then $q_i(x_t) \ge 1/2$,
 and if $y_t \neq i$ then $q_i(x_t) \le -1/2$, and
\begin{align*}
  \norm{q_i} & \le (4K-5) 2^{K-1} \\
  & \qquad \cdot \left(2^{s} r(K-1) (4s+2)^2 \right)^{(s+1/2)r(K-1)} \; .
\end{align*}
By \autoref{lemma:norm-bound}, there exists $c'_i \in \ell_2$ such that
$\ip{c'_i}{\phi(x)} = p_i(x)$, and
\begin{align*}
  \norm{c'_i}_{\ell_2} & \le (4K-5) 2^{K-1} \\
  & \qquad \cdot \left(2^{s+1} r(K-1) (4s+2)^2 \right)^{(s+1/2)r(K-1)} \; .
\end{align*}
Define vectors $u_i' \in \ell_2$ as
\begin{align*}
u_i' & = \frac{c_i'  \cdot \left(2^{s+1} r(K-1) (4s+2)^2 \right)^{-(s+1/2)r(K-1)}}{\sqrt{K} (4K-5) 2^{K-1}} \; .
\end{align*}
Then,
$\norm{u_1'}^2 + \norm{u_2'}^2 + \dots + \norm{u_K'}^2 \le 1$.
Furthermore, for all $t=1,2,\dots,T$,
$\ip{u'_{y_t}}{x_t} \ge \gamma_2$
and for all $j \in \{1,2,\dots,K\} \setminus \{y_t\}$,
$\ip{u'_j}{x_t} \le - \gamma_2$. In other words,
$(\phi(x_1), y_1), (\phi(x_2), y_2), \dots, (\phi(x_T), y_T)$ are
strongly linearly separable with margin $\gamma_2 = \max\{\gamma_1, \gamma_2\}$.
%The proof for the case of $\gamma_1 < \gamma_2$
%is symmetric.

In summary, the examples are strongly
linearly separable with margin $\gamma' = \max\{\gamma_1, \gamma_2\}$.
Finally, observe that for any $t=1,2,\dots,T$,
\[
k(x_t,x_t) = \frac{1}{1 - \frac{1}{2} \norm{x_t}^2} \le 2 \; .
\qedhere
\]
\end{proof}
