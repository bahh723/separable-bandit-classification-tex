\section{NP-hardness of the weak labeling problem}

Any algorithm for the bandit setting collects information in the form of so
called \emph{strongly labeled} and \emph{weakly labeled} examples.
Strongly-labeled examples are those for which we know the class label. Weakly
labeled example is an example for which we know that class label can be anything
except for a particular one class.

A natural strategy for each round is to find vectors $w_1, w_2, \dots, w_K$ that
linearly separate the examples seen in the previous rounds and use the vectors
to predict the label in the next round. More precisely, we want to find both the
vectors $w_1, w_2, \dots, w_K$ and label for each example consistent with its
weak and/or strong labels such that $w_1, w_2, \dots, w_K$ linearly separate the
labeled examples. We show this problem is NP-hard even for $K=3$.

Clearly, the problem is at least as hard as the decision version of the problem
where the goal is to determine if such vectors and labeling exist. We show that
this problem is NP-complete.

Formally, the input to the decision problem are positive integers $d, T, K$ and
a sequence $(x_1, y_1), (x_2, y_2), \dots, (x_T, y_T) \in \{0,1\}^d \times
\{1,2,\dots,K, \overline{1}, \overline{2}, \dots, \overline{K}\}$ of strongly
and weakly labeled examples. The numbers $1,2,\dots,K$ correspond to strong
labels and the symbols $\overline{1}, \overline{2}, \dots, \overline{K}$
correspond to weak labels. We adopt the convention that $\overline{\overline{i}}
= i$ for any positive integer $i$. The goal is to determine whether or not there
exist vectors $w_1, w_2, \dots, w_K \in \R^d$ that satisfy the following
conditions for all $t=1,2,\dots,T$,
\begin{align*}
y_t \in \{1,2,\dots,K\} \qquad & \Longrightarrow \qquad & \forall i \in \{1,2,\dots,K\} \setminus \{y_t\} \qquad \ip{w_{y_t}}{x_t} & > \ip{w_i}{x_t} \; , \\
y_t \in \{\overline{1}, \overline{2},\dots, \overline{K}\} \qquad & \Longrightarrow & \qquad \exists i \in \{1,2,\dots,K\} \qquad \ip{w_i}{x_t} & > \ip{w_{\overline{y_t}}}{x_t} \; .
\end{align*}

TODO
