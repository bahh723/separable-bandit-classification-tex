\section{NP-hardness of the weak labeling problem}

Any algorithm for the bandit setting collects information in the form of so
called \emph{strongly labeled} and \emph{weakly labeled} examples.
Strongly-labeled examples are those for which we know the class label. Weakly
labeled example is an example for which we know that class label can be anything
except for a particular one class.

A natural strategy for each round is to find vectors $w_1, w_2, \dots, w_K$ that
linearly separate the examples seen in the previous rounds and use the vectors
to predict the label in the next round. More precisely, we want to find both the
vectors $w_1, w_2, \dots, w_K$ and label for each example consistent with its
weak and/or strong labels such that $w_1, w_2, \dots, w_K$ linearly separate the
labeled examples. We show this problem is NP-hard even for $K=3$.

Clearly, the problem is at least as hard as the decision version of the problem
where the goal is to determine if such vectors and labeling exist. We show that
this problem is NP-complete.

Formally, the weak labeling problem can be described as below: 
\begin{figure}[H]
\begin{framed}
\begin{center}
    \textbf{Weak Labeling}
\end{center}
\textbf{Given: } A set of feature-label pairs: $\{(x_t, y_t)\}_{t=1}^T \subseteq \{0,1\}^d \times \{1,2,\ldots, K, \overline{1}, \overline{2}, \ldots, \overline{K}\}$. \\
\textbf{Question: } Do there exist $ w_1, \ldots, w_K \in \R^d$ such that for all $t=1, \ldots, T$,  
\begin{align*}
y_t \in \{1,2,\dots,K\} \qquad & \Longrightarrow \qquad & \forall i \in \{1,2,\dots,K\} \setminus \{y_t\} \qquad & \ip{w_{y_t}}{x_t}  > \ip{w_i}{x_t} \; , \\
y_t \in \{\overline{1}, \overline{2},\dots, \overline{K}\} \qquad & \Longrightarrow & \qquad \exists i \in \{1,2,\dots,K\} \qquad & \ip{w_i}{x_t} > \ip{w_{\overline{y_t}}}{x_t} \; .
\end{align*}
\end{framed}
\end{figure}


%Formally, the input to the decision problem are positive integers $d, T, K$ and
%a sequence $(x_1, y_1), (x_2, y_2), \dots, (x_T, y_T) \in \{0,1\}^d \times
%\{1,2,\dots,K, \overline{1}, \overline{2}, \dots, \overline{K}\}$ of strongly
%and weakly labeled examples. The numbers $1,2,\dots,K$ correspond to strong
%labels and the symbols $\overline{1}, \overline{2}, \dots, \overline{K}$
%correspond to weak labels. We adopt the convention that $\overline{\overline{i}}
%= i$ for any positive integer $i$. The goal is to determine whether or not there
%exist vectors $w_1, w_2, \dots, w_K \in \R^d$ that satisfy the following
%conditions for all $t=1,2,\dots,T$,
%\begin{align*}
%y_t \in \{1,2,\dots,K\} \qquad & \Longrightarrow \qquad & \forall i \in \{1,2,\dots,K\} \setminus \{y_t\} \qquad \ip{w_{y_t}}{x_t} & > \ip{w_i}{x_t} \; , \\
%y_t \in \{\overline{1}, \overline{2},\dots, \overline{K}\} \qquad & \Longrightarrow & \qquad \exists i \in \{1,2,\dots,K\} \qquad \ip{w_i}{x_t} & > \ip{w_{\overline{y_t}}}{x_t} \; .
%\end{align*}

The hardness proof is based on a reduction from the set splitting problem, which is proven to be NP-complete by Lovasz \cite{garey1979computers}, to our weak labeling problem. The reduction is adapted from \cite{blum1989training}. 
\begin{figure}[h]
\begin{framed}
\begin{center}
    \textbf{Set Splitting}
\end{center}
\textbf{Given: } A finite set $S$ and a collection $C$ of subsets $c_i$ of $S$. \\
\textbf{Question: } Do there exist disjoint sets $S_1$ and $S_2$ such that $S_1 \cup S_2 = S$ and $\forall i, c_i\not\subseteq S_1$ or $c_i\not\subseteq S_2$? 
\end{framed}
\end{figure}

Below we show the reduction. Suppose we are given a set splitting problem with inputs: 
\begin{align*}
    &S = \{1, 2, \ldots, N\}, \\
    &C = \{c_1, \ldots, c_M\}, \\ 
    &\forall i, c_i \subseteq S. 
\end{align*}

Then we create the weak labeling instance as follows. Let $d=N+1$ and $K=3$. Define $\zero$ as the zero vector $(0,\ldots, 0)\in \R^N$ and $\e_i$ as the $i$-th standard vector $(0,\ldots, 1, \ldots, 0)\in \R^N$). Then we include all the following feature-label pairs: 
\begin{itemize}
    \item Type 1: $(x,y)=((\zero,1), 3)$,
    \item Type 2: $(x,y)=((\e_i,1), \overline{3})$, $\forall i\in [N]$,
    \item Type 3: $(x,y)=\left(\left(\sum_{i\in c_j} \e_i, 1\right), 3\right)$, $\forall j\in[M]$.  
\end{itemize}

For example, if we have $S=\{1,2,3\}$, $C=\{c_1, c_2\}$, $c_1 = \{1,2\}$, $c_2=\{2,3\}$, then we create the weak labeling sample set as: 
\begin{align*}
    \{((0,0,0,1),3), ((1,0,0,1),\overline{3}), ((0,1,0,1),\overline{3}), ((0,0,1,1),\overline{3}), ((1,1,0,1),3), ((0,1,1,1),3)\}.
\end{align*}
The following lemma shows that answering this weak labeling problem is equivalent to answering the original set splitting problem. 
\begin{lemma}
Suppose we are given a set splitting problem, and we construct a weak labeling problem as described above. Then the set splitting problem has a solution $\Longleftrightarrow$ the constructed weak labeling problem has a solution. 
\end{lemma}
\begin{proof}
 $(\Longrightarrow)$ Let $S_1, S_2$ be the solution of the set splitting problem. Define
 \begin{align*}
     w_1 = \left(a_1, a_2, \cdots, a_N, -\frac{1}{2}\right), 
 \end{align*}
 where $\forall i\in[N]$, $a_i=1$ if $i\in S_1$ and $a_i=-N$ if $i\notin S_1$. 
 Similarly, define
 \begin{align*}
     w_2 = \left(b_1, b_2, \cdots, b_N, -\frac{1}{2}\right), 
 \end{align*}
 where $\forall i\in[N]$, $b_i=1$ if $i\in S_2$ and $b_i=-N$ if $i\notin S_2$. 
 Finally, define
 \begin{align*}
     w_3 = (0,0,\cdots, 0), 
 \end{align*}
 the zero vector. To see this is a solution for the weak labeling problem, we verify separately for Type 1-3 samples defined above. For Type 1 sample, we have
 \begin{align*}
     \ip{w_3}{x} = 0 > -\frac{1}{2} = \ip{w_1}{x}=\ip{w_2}{x}. 
 \end{align*}
 For a Type 2 sample that corresponds to index $i$, we have either $i\in S_1$ or $i\in S_2$ because $S_1\cup S_2 = [N]$ is guaranteed. Thus, either $a_i=1$ or $b_i=1$. If $a_i=1$ is the case, then
 \begin{align*}
     \ip{w_1}{x}=a_i-\frac{1}{2}=\frac{1}{2} > 0 = \ip{w_3}{x}; 
 \end{align*}
 similarly if $b_i=1$, we have $\ip{w_2}{x}>\ip{w_3}{x}$. \\
 For a Type 3 sample that corresponds to index $j$, Since $c_j \not\subset S_1$, there exists some $i'\in c_j$ and $i'\notin S_1$. Thus we have $x_{i'}=1$, $a_{i'}=-N$, and therefore
 \begin{align*}
     \ip{w_1}{x}= a_{i'}x_{i'} + \sum_{i\in [N]\backslash\{i'\}} a_ix_i - \frac{1}{2} \leq -N + (N-1)-\frac{1}{2} < 0 = \ip{w_3}{x}. 
 \end{align*}
 Because $c_j \not\subset S_2$ also holds, we also have $\ip{w_2}{x}<\ip{w_3}{x}$. This direction is therefore proved. \\
 \ \\
 $(\Longleftarrow)$ Given the solution $w_1, w_2, w_3$ of the weak labeling problem, we define 
 \begin{align*}
     S_1 = \left\{i\in [N] \Big| \ip{w_1-w_3}{(\e_i, 1)} > 0 \right\}, \\
     S_2 = \left\{i\in [N] \Big| \ip{w_2-w_3}{(\e_i, 1)} > 0 \right\}.
 \end{align*}
 It is not hard to see $S_1\cup S_2 = [N]$. Note that $(\e_i, 1)$ is the feature vector for Type 2 samples. Because Type 2 samples all have label $\overline{3}$, for any $i\in [N]$, one of the following must hold: $\ip{w_1-w_3}{(\e_i, 1)}>0$ or $\ip{w_2-w_3}{(\e_i, 1)}>0$. Thus $i\in S_1$ or $i\in S_2$. 
 
 Now we show $\forall j$, $c_j \not\subset S_1$ and $c_j \not\subset S_2$ by contraction. Without loss of generality, suppose there exists some $j$ such that $c_j \subset S_1$. By our definition of $S_1$, we have $\ip{w_1-w_3}{(\e_i, 1)} > 0$ for all $i\in c_j$. Therefore, 
 \begin{align*}
     \sum_{i\in c_j}\ip{w_1-w_3}{\left(\e_i, 1\right)} = \ip{w_1-w_3}{\left(\sum_{i\in c_j}\e_i, |c_j|\right)} > 0. 
 \end{align*}
 Because Type 1 sample has label $3$, we also have
 \begin{align*}
     \ip{w_1-w_3}{\left(\zero, 1\right)} < 0. 
 \end{align*}
 Combining the above two inequalities, we get
 \begin{align*}
     \ip{w_1-w_3}{\left(\sum_{i\in c_j}\e_i, 1\right)} = \ip{w_1-w_3}{\left(\sum_{i\in c_j}\e_i, |c_j|\right)} - (|c_j|-1)\ip{w_1-w_3}{\left(\zero, 1\right)}>0.  
 \end{align*}
 Note that $\left(\sum_{i\in c_j}\e_i, 1\right)$ is a feature vector for Type 3 samples. Thus the above inequality contradicts that Type 3 samples have label 3. Therefore, $c_j \not\subset S_1$. Similarly we can prove $c_j \not\subset S_2$. 
\end{proof}