\documentclass{article}

% if you need to pass options to natbib, use, e.g.:
% \PassOptionsToPackage{numbers, compress}{natbib}
% before loading nips_2017
%
% to avoid loading the natbib package, add option nonatbib:
% \usepackage[nonatbib]{nips_2017}

%\usepackage{nips_2017}

% to compile a camera-ready version, add the [final] option, e.g.:
 \usepackage[final]{nips_2017}

%\usepackage[utf8]{inputenc} % allow utf-8 input
\usepackage[T1]{fontenc}    % use 8-bit T1 fonts
\usepackage{hyperref}       % hyperlinks
\usepackage{url}            % simple URL typesetting
\usepackage{booktabs}       % professional-quality tables
\usepackage{amsfonts}       % blackboard math symbols
\usepackage{nicefrac}       % compact symbols for 1/2, etc.
\usepackage{microtype}      % microtypography
% For citations
\usepackage{natbib}
\usepackage{amsthm}
% For algorithms
%\usepackage[algcompatible]{algpseudocode}
%\usepackage{algorithm}
%\usepackage{algorithmicx}
%\usepackage[noend]{algcompatible}
\usepackage[vlined,linesnumbered,ruled]{algorithm2e}
\usepackage{bbm}
\usepackage{amsfonts}
\usepackage{amsmath}           
\usepackage{mathtools}    
\usepackage{amsopn}
\usepackage{amssymb}
\usepackage{graphicx}
\usepackage{enumerate}
%%%%%%%%%%%%%%%%%% toc 
\usepackage{scrwfile}
\TOCclone[\contentsname~(\appendixname)]{toc}{atoc}
\newcommand\StartAppendixEntries{}
\AfterTOCHead[toc]{%
  \renewcommand\StartAppendixEntries{\value{tocdepth}=-10000\relax}%
}
\AfterTOCHead[atoc]{%
  \edef\maintocdepth{\the\value{tocdepth}}%
  \value{tocdepth}=-10000\relax%
  \renewcommand\StartAppendixEntries{\value{tocdepth}=\maintocdepth\relax}%
}
\newcommand*\appendixwithtoc{%
  \cleardoublepage
  \appendix
  \addtocontents{toc}{\protect\StartAppendixEntries}
  \listofatoc
}
\usepackage{blindtext}
\usepackage{color}
%%%%%%%%%%%%%%%%%% toc

%%%%%%%%%%%%%%%% HEADER 
\newcommand{\EG}{\textsc{Epoch-Greedy}\xspace}
\newcommand{\EPG}{\textsc{$\epsilon$-Greedy}\xspace}
\newcommand{\minimonster}{\textsc{ILOVETOCONBANDITS}\xspace}
\newcommand{\ILTCB}{\textsc{ILTCB}\xspace}
\newcommand{\AdaEG}{\textsc{Ada-Greedy}\xspace}
\newcommand{\AdaILTCB}{\textsc{Ada-ILTCB}\xspace}
\newcommand{\AdaPE}{\textsc{Ada-PE}\xspace}
\newcommand{\corral}{\textsc{Corral}\xspace}
\newcommand{\bistro}{\textsc{BISTRO+}\xspace}
\newcommand{\base}[1]{{{\cal{B}}_{#1}}}
\newcommand{\scale}{\rho}
\newcommand{\FS}{\text{FS}}

\newcommand{\calA}{{\mathcal{A}}}
\newcommand{\calX}{{\mathcal{X}}}
\newcommand{\calS}{{\mathcal{S}}}
\newcommand{\calI}{{\mathcal{I}}}
\newcommand{\calJ}{{\mathcal{J}}}
\newcommand{\calK}{{\mathcal{K}}}
\newcommand{\calD}{{\mathcal{D}}}
\newcommand{\calE}{{\mathcal{E}}}
\newcommand{\calR}{{\mathcal{R}}}
\newcommand{\calT}{{\mathcal{T}}}
\newcommand{\calH}{\mathcal{H}}
\newcommand{\avgR}{\wh{\cal{R}}}
\newcommand{\ips}{\wh{r}}
\newcommand{\whpi}{\wh{\pi}}
\newcommand{\whE}{\wh{\E}}
\newcommand{\whV}{\wh{V}}
\newcommand{\Reg}{\text{\rm Reg}}
\newcommand{\whReg}{\wh{\text{\rm Reg}}}
\newcommand{\flg}{\text{\rm flag}}
\newcommand{\one}{\boldsymbol{1}}
\newcommand{\var}{\Delta}
\newcommand{\p}{\prime}
\newcommand{\nb}{\nabla}
\newcommand{\e}{\mathbf{e}}

\DeclareMathOperator*{\arginf}{arginf}
\DeclareMathOperator*{\argsup}{argsup}
\DeclareMathOperator*{\range}{range}
\DeclareMathOperator*{\mydet}{det_{+}}
\DeclarePairedDelimiter\abs{\lvert}{\rvert}
\DeclarePairedDelimiter\bigabs{\big\lvert}{\big\rvert}
\DeclarePairedDelimiter\ceil{\lceil}{\rceil}
\DeclarePairedDelimiter\floor{\lfloor}{\rfloor}
\DeclarePairedDelimiter\bigceil{\big\lceil}{\big\rceil}
\DeclarePairedDelimiter\bigfloor{\big\lfloor}{\big\rfloor}

\newcommand{\field}[1]{\mathbb{#1}}
\newcommand{\fY}{\field{Y}}
\newcommand{\fX}{\field{X}}
\newcommand{\fH}{\field{H}}
\newcommand{\fR}{\field{R}}
\newcommand{\fN}{\field{N}}
\newcommand{\E}{\field{E}}
\newcommand{\err}{\text{err}}

\newcommand{\theset}[2]{ \left\{ {#1} \,:\, {#2} \right\} }
\newcommand{\inner}[1]{ \left\langle {#1} \right\rangle }
\newcommand{\Ind}[1]{ \field{I}_{\{{#1}\}} }
\newcommand{\eye}[1]{ \boldsymbol{I}_{#1} }
\newcommand{\norm}[1]{\left\|{#1}\right\|}
%\newcommand{\trace}[1]{\text{tr}\left({#1}\right)}
\newcommand{\trace}[1]{\textsc{tr}({#1})}
\newcommand{\diag}[1]{\mathrm{diag}\!\left\{{#1}\right\}}

\newcommand{\defeq}{\stackrel{\rm def}{=}}
\newcommand{\sgn}{\mbox{\sc sgn}}
\newcommand{\scI}{\mathcal{I}}
\newcommand{\scO}{\mathcal{O}}
\newcommand{\scN}{\mathcal{N}}

\newcommand{\dt}{\displaystyle}
\renewcommand{\ss}{\subseteq}
\newcommand{\wh}{\widehat}
\newcommand{\wt}{\widetilde}
\newcommand{\ve}{\varepsilon}
\newcommand{\hlambda}{\wh{\lambda}}
\newcommand{\yhat}{\wh{y}}

\newcommand{\hDelta}{\wh{\Delta}}
\newcommand{\hdelta}{\wh{\delta}}
\newcommand{\spin}{\{-1,+1\}}
%\newcommand{\calS}{\mathcal{S}}
\newcommand{\calC}{\mathcal{C}}

%%%%%%%%%%%%%%%%%%%%%%%%

\newtheorem{assumption}{Assumption}
\newtheorem{theorem}{Theorem}
\newtheorem{lemma}[theorem]{Lemma}
\newtheorem{corollary}[theorem]{Corollary}
\newtheorem{proposition}[theorem]{Proposition}
\newtheorem{definition}[theorem]{Definition}
\newtheorem{remark}[theorem]{Remark}
\DeclareMathOperator{\val}{val}
\DeclareMathOperator{\spa}{sp}
\DeclareMathOperator{\solve}{solve}
\allowdisplaybreaks

% Packages hyperref and algorithmic misbehave sometimes.  We can fix
% this with the following command.
%\newcommand{\theHalgorithm}{\arabic{algorithm}}

\usepackage{amsmath}
\DeclareMathOperator*{\argmin}{\arg\!\min}
\DeclareMathOperator*{\argmax}{\arg\!\max}

\title{Bandit Multiclass Classification}

%\author{}

\begin{document}
% \nipsfinalcopy is no longer used

\maketitle
	
%#############################################################################
%    Introduction
%#############################################################################

\section{A Regret Lower Bound for A Certain Type of Algorithms}
In this section, we try to construct an error lower bound for a certain type of algorithms. This type of algorithms does not make update when it makes a wrong prediction. For simplicity, we only consider binary classification.  More formally, the algorithms we consider satisfy the following assumption. 

\begin{assumption}[Algorithm]
Let $p_t(x)$ be the algorithm's probability of predicting class $1$ (recall we consider binary classification) at round $t$ if it receives the feature vector $x\in \calX \subset \mathbb{R}^d$. We assume $p_t(\cdot)$ is totally determined by all previous \textbf{correct} examples. In other words, $p_t(\cdot)$ is determined by the tuple $((x_{\tau_1}, y_{\tau_1}), \ldots, (x_{\tau_N}, y_{\tau_N}))$ where $1\leq \tau_1 < \tau_2 \ldots < \tau_N < t$ are the rounds that the learner makes correct prediction. 
\end{assumption}

\begin{assumption}[linearly separable with a margin]
We assume the samples are linearly separable with margin $\gamma$ (i.e., any two points with different labels have distance no less than $\gamma$). 
\end{assumption}

\begin{definition}[Free space]
The free space at time $t$ is the set of points whose label is still underdetermined given $(x_1, y_1), \ldots, (x_{t-1},y_{t-1})$. For example, the $\gamma$-ball centered around any already presented point is excluded from the free space. Denote the free space at time $t$ by $\FS_t$.  
\end{definition}
The free space's definition simply means that at time $t$, the adversary can pick any point $x_t$ in $\FS_t$ and assign the label $y_t$ to either $1$ or $2$ without violating the linearly separable and the $\gamma$-margin assumption. 

Below we present the Adversary's strategy of constructing $(x_t, y_t)$. 

\begin{algorithm}[H]
\caption{Adversary's strategy}
Pick $x_1$ randomly from $\calX$, and let $y_1=1$. \\
\For{$t=2, \ldots, T$}{
    \If{$\tilde{y}_{t-1} \neq y_{t-1}$}{
        Let $(x_t, y_t)=(x_{t-1}, y_{t-1})$
    }
    \ElseIf{$\FS_{t}$ is not empty}{
        Pick $x_t\in \FS_{t}$. Because of this $x_t$, the free space's volume is reduced. We denote the reduction amount by $V_t=v(\FS_{t+1})-v(\FS_{t})\leq V$. (i.e., $V$ is a global upper bound of $V_t$) \\
        If $p_t(x_t)\geq 1-\max\left\{\sqrt{V},\frac{1}{\sqrt{T}}\right\}$, then label $y_t=2$; otherwise, label $y_t=1$. 
    }
    \Else{
        Randomly assign $(x_t,y_t)$ with some value that does not violate the assumption. 
    }
}
\end{algorithm}

%\textbf{Analysis}. We assume that in the ``else'' case in the above algorithm, we can always pick $x_t$ such that $V_t\leq V$ (we will show how to pick such $x_t$ later). 

\begin{definition}[history]
Let $\calH_t$ be the history before time $t$: $\calH_t=\{ (x_s, y_s, \tilde{y}_s) \}_{s=1}^{t-1}$. We use $\E_t[\cdot]$ to denote $\E[\cdot | \calH_t]$. 
\end{definition}

\begin{lemma}
If $\exists t$ such $p_t(x_t)\geq 1-\max\left\{ \sqrt{V}, \frac{1}{\sqrt{T}} \right\}$, then $\E_t\left[ \sum_{s=t}^T  \one[\tilde{y}_s\neq y_s] \right] \geq \Omega\left(\min\left\{\frac{1}{\sqrt{V}},\sqrt{T}\right\}\right)$.
\end{lemma}
\begin{proof}
By of the condition and the adversary strategy, we have $y_t=2$. Therefore, the learner will predict the true label with probability $\leq \max \left\{ \sqrt{V}, \frac{1}{\sqrt{T}} \right\}$. And note that if the learner predicts incorrectly at time $t$, then at time $t+1$ the feature vector remains the same $(x_{t+1}=x_t)$ , and the learner's probability of prediction also remains the same $(p_{t+1}(\cdot)=p_t(\cdot))$. Therefore, the expected number of mistakes before the first correct guess is (roughly) larger than  $\frac{1}{\max\left\{\sqrt{V}, \frac{1}{\sqrt{T}}\right\}}=\min\left\{\frac{1}{\sqrt{V}}, \sqrt{T}\right\}$. 
\end{proof}

\begin{lemma}
If $\forall s$, $p_s(x_s)\leq 1-\max\left\{\sqrt{V}, \frac{1}{\sqrt{T}}\right\}$, then $\sum_{s=1}^T \E_s[\one[\tilde{y}_s\neq y_s]] = \Omega\left(\min\left\{\sqrt{T}, \frac{1}{\sqrt{V}}\right\}\right)$.
\end{lemma}

\begin{proof}
By the condition and the adversary strategy, we know that the probability of error is larger than $\max\left\{\sqrt{V},\frac{1}{\sqrt{T}}\right\}$ at all time $t$ \textit{before the free space is used up}. Since each time the free space only reduces by $V$, in the first $\frac{1}{V}$ rounds (assume the total volume is $1$), the free space is still all available. Therefore, 
\begin{align*}
    \sum_{s=1}^T \E_s[\one[\tilde{y}_s\neq y_s]] \geq \min\left\{ T,\frac{1}{V} \right\}\times \max\left\{ \sqrt{V}, \frac{1}{\sqrt{T}} \right\}=\min\left\{\sqrt{T}, \frac{1}{\sqrt{V}}\right\}. 
\end{align*}
\end{proof}

\textbf{Discussion}. We believe that $V$ can be in the order of $\mathcal{O}\left(\gamma^d\right)$ (some weaker thing like $\mathcal{O}\left(\gamma^{d/2}\right)$ is also acceptable). Basically, we need to figure out the following question: can we construct a sequence of points $\{x_1,x_2, \ldots\}$ , such that for every $t$, $x_t$'s distance with $\text{conichull}\{x_1,\ldots, x_{t-1}\}$ is larger than $\gamma$, but the volume difference $v(\text{conichull}\{x_1,\ldots, x_{t}\}) - v(\text{conichull}\{x_1,\ldots, x_{t-1}\})$ is upper bounded by $V$?

This lower bound does not rule out those algorithms that change its probability vector based on the $\textbf{count}$ of consecutive errors. This type of algorithm may still easy to be implemented (like QBC). Also it does not rule out the banditron algorithm. 

\section{Biased Halving: Trading Error with Complexity}
\begin{algorithm}[H]
\caption{Banditron}
\textbf{Define}: $\Omega=\{W\in \mathbb{R}^{Kd}: \|\e_i^\top W\|_2\leq D\}$.\\
For a set $S$ of $W$'s, $S(i|x)$ is the subset of $S$ that outputs class $i$ given feature vector $x$, i.e., $S(i|x)=\{W\in S: (Wx)_i \geq (Wx)_j \ \forall j\}$\\
$|S|$ denotes the volumn of $S$. \\
\textbf{parameter}: $\alpha \in (0,1)$ \\
$\Omega_1=\Omega$.  \\
\For{$ t=1,\ldots, T $}{
\If{$\argmax_i \frac{|\Omega_t(i|x_t)|}{|\Omega_t|} \geq 1-\alpha$}{ \label{line:case1}
        Let $\tilde{y}_t=i$.\\ 
        \If{$\tilde{y}_t \neq y_t$}{
            $\Omega_{t+1}=\Omega_t \backslash \Omega_t(\tilde{y}_t|x_t)$. \label{line:cut}
        }
    }
\label{line:case2}\Else{
        Let $\tilde{y}_t\sim \text{unif}([K])$. \\
        \If{$\tilde{y}_t=y_t$}{
            $\Omega_{t+1}=\Omega_t(\tilde{y}_t|x_t)$. \label{line:cut2}
        }
    }
}
\end{algorithm}

\textbf{Conjecture(should be true)}: If the volume of $\Omega_t$ becomes smaller than $\frac{|\Omega|}{N}$, then the algorithm won't make any error anymore. $N$ should be in the order of $\Theta(\frac{1}{\gamma^{Kd}})$.  \\
\textbf{Rough analysis:} \\
Each time the algorithm makes an error in Line~\ref{line:case1}, the volume becomes $\alpha$ times the original volume. So the algorithm will not make more than $ \frac{\ln N}{\ln \frac{1}{\alpha}} $ mistakes in this case. 

In the case of Line~\ref{line:case2}, $K\ln \frac{1}{\delta}$ errors will accompany with a $(1-\alpha)$-factor shrinkage in the volume. Therefore, the number of errors occurred in this case is upper bounded by $ \frac{K\ln\frac{1}{\delta}\ln N}{\ln \frac{1}{1-\alpha}}\leq \frac{K\ln\frac{1}{\delta}\ln N}{\alpha} $.

Now we discuss about the complexity. The main issue is how to maintain $\Omega_t$. Each time the algorithm enters Line~\ref{line:cut}, $\Omega_t$ becomes more and more fragmented. But if $\Omega_t$ can be maintained with $M$ convex cones, then $\Omega_{t+1}$ can be maintained with $(K-1)M\leq KM$ convex cones. And we assume each cone's volume can be computed in \text{poly}($T$) time. Each time the algorithm enters Line~\ref{line:cut2}, the number of convex cones does not increase. 

By the above discussion, there will be no more than $K^{\frac{\ln N}{\ln \frac{1}{\alpha}}}$ convex cones to maintain. And the error bound is in the order of $\frac{K\ln\frac{1}{\delta}\ln N}{\alpha}$ for some $\alpha<\frac{1}{2}$. Let's try to balance the number of errors and computational complexity. Let
\begin{align*}
    &K^{\frac{\ln N}{\ln \frac{1}{\alpha}}}\approx\frac{K\ln\frac{1}{\delta}\ln N}{\alpha}\\
    \Rightarrow &~ \frac{\ln N}{\ln\frac{1}{\alpha}} \ln K \approx  \ln\left(K\ln\frac{1}{\delta}\ln N\right) + \ln\frac{1}{\alpha}\\
    \Rightarrow &~ \text{pick\ }\ln\frac{1}{\alpha}=\sqrt{\ln N}.
\end{align*}
Thus the computational complexity is in the order of $K^{\sqrt{\ln N}}\times \text{poly}(T)=K^{\sqrt{Kd\ln\frac{1}{\delta}}}$. The error bound is $\mathcal{O}\left( e^{\sqrt{Kd\ln \frac{1}{\delta}}} K^2 d \ln\frac{1}{\delta}\ln\frac{1}{\gamma} \right)$.

Another viewpoint: let $\frac{1}{\alpha}=K^\beta$, then the complexity is $\left(\frac{1}{\gamma}\right)^{\frac{Kd}{\beta}}\times \text{poly}(T)$ and the error bound is $K^{\beta+1}\ln\frac{1}{\delta}\ln N$. 

\section{How hard it is to use the feedback only from wrong guesses?}
The halving algorithm can actually run if we can do ``uniform sampling'' over the version space. But it is even unknown whether we can efficiently pick a model from the version space. The problem is that we get a lot of feedback in the form of ``feature $x_t$ does not belong to class $\tilde{y}_t$'', which we don't know how to use. 

The following is just an attempt (not successful but might be interesting...) to say that it might be not easy to figure out a model if the learner is only presented with this kind of ``error message''. 

The problem is formulated as follows. 
Given $N$ points in a row, each one with a class $c_i\in [K]$, $\forall i\in [N]$. We call these $N$ points \textit{separable} if the following statement holds: 
\begin{center}
    If $c_i=c_j$ for some $i\leq j$, then $c_i=c_{i+1}=\cdots=c_j$.
\end{center}
For example, if $N=5, K=3$, then $(c_1,c_2, c_3, c_4,c_5)=(3,3,1,1,1)$ is separable, but $(c_1,c_2,c_3,c_4,c_5)=(2,1,2,2,2)$ is not. 
\ \\
\ \\
Now you have $N$ conditions, in which the $i$-th condition only says something like ``$c_i\neq k$'' for some $k$. \\
(1) Can you efficiently decide whether there exists an assignment of $(c_1,\ldots,c_N)$ such that these $N$ points are separable and satisfy all the conditions? \\
(2) If it is guaranteed that there are separable solutions, can you efficiently find one of them?\\ (Efficient: the complexity is polynomial in $N$ and $K$)
\ \\
\ \\
\textbf{Example 1}. \\
$N=5, K=3$: \\
$c_1\neq 1$ \\
$c_2\neq 2$ \\
$c_3\neq 3$ \\
$c_4\neq 1$ \\
$c_5\neq 2$ \\
$\Rightarrow (c_1,c_2,c_3,c_4,c_5)=(2,1,1,3,3)$ or $(3,3,2,2,1)$ are separable solutions. \\
\ \\
\textbf{Example 2}. \\
$N=7, K=3$: \\
$c_1\neq 1$ \\
$c_2\neq 2$ \\
$c_3\neq 3$ \\
$c_4\neq 1$ \\
$c_5\neq 2$ \\
$c_6\neq 3$ \\
$c_7\neq 1$ \\
$\Rightarrow$ There is no separable solution. \\

It turns out this specific 1-dimensional problem is equivalent to check whether there is a \textbf{missing permutation} of $[K]$ / find a missing permutation in the given subsequence. In example 1, the existed permutations are $(1,2,3), (1,3,2), (2,3,1), (3,1,2)$, the missing ones are $(2,1,3)$ and $(3,2,1)$.  So the solutions can be $(2,1,3)$ or $(3,2,1)$ (with some repetition). In Example 2, all permutations are as subsequences, so there is no solution. 

We can prove this equivalence considering two directions: \\
(1) If there is a solution with the class labels following a permutation, then that permutation cannot be a subsequence of the given sequence. \\
(2) If there is a missing permutation in the given sequence, then there is a class assignment that follows this permutation.\\ 
They should be straightforward by trying some examples. 


\section{Cone Algorithm}
\begin{algorithm}[H]
\caption{Banditron}
\textbf{definition}: $K\triangleq$ number of classes, $\gamma\triangleq$  margin\\
\textbf{Initialize}: $\calS_1=\cdots=\calS_K=\phi$ (empty set) \\
\For{$t=1, \ldots, T$}{
    Receive $x_t\in \mathbb{R}^d$. \\
    Define the cone $\calC_i = \left\{x\in\mathbb{R}^d: x=\sum_{j=1}^{|\calS_i|} \alpha_jy_j, \text{\ where\ } y_j\in \calS_i, \alpha_j\geq 0 \right\}$ \\
    (that is, $\calC_i$ is the conic hull of $\calS_i$) \\
    Check whether $x_t$ belongs to, or has distance smaller than $\gamma$, to one of $\calC_1, \ldots, \calC_K$. \\
    If so, classify $x_t$ to the corresponding class (say class $i$), and let $\calS_i\leftarrow \calS_i \cup \{x_t\}$. {\color{red} This prediction will be correct for sure by our margin assumption.}\\
    If not, let $\tilde{y}_t\sim \text{unif}([K])$ and predict $\tilde{y}_t$. If $\tilde{y}_t=y_t$, then $\calS_{y_t}\leftarrow \calS_{y_t} \cup \{x_t\}$. 
}
\end{algorithm}




\section{One-against-all Perceptron}
Same as Chicheng's writeup's Section 5: Fixed-Threshold Perceptron. 

\section{Gradient Descent {\color{red} [TODO]}}

%\textbf{Observation}. Alekh's paper's results can be generalized to losses with $\Phi_t$ a strongly and smoothly convex function. 
\begin{assumption}
$\|x_t\|_2^2\leq 1$. There is a $W^*\in\mathcal{W}$ such that $\ell_t(W^*)\leq 0$ for all $t$ ($\ell_t$ and $\mathcal{W}$ are defined below).  
\end{assumption}

\begin{algorithm}[H]
\caption{Banditron}
\textbf{Input}: $D\geq 2$, $\epsilon$ (picked in a later lemma). \\
\textbf{Definition}: 
\begin{align*}
    \ell_t(W) &\triangleq [1-(Wx_t)_{y_t} + \max_{r\neq y_t}(Wx_t)_r]_+^2 \ \ \ \ \text{(squared hinge loss)} \\
    & = \Phi_t(Wx_t), 
    %\Phi_t(z) \triangleq \sum_{i\neq y_t} \psi(z_{y_t}-z_i).
\end{align*}
where $\Phi_t(z)\triangleq [1-\e_{y_t}^\top z + \max_{r\neq y_t} \e_r^\top z]_+^2$. \\
Also, define $\mathcal{W}=\{W\in \mathbb{R}^{K\times d}: \|\e_i^\top W\|_2\leq D \text{\ for all }i\in[K]\}$. \\
\textbf{Initialization}: $W_1=0, M_1=I$.\\ 
\For{$t=1, \ldots, T$}{
   Observe $x_t$. \\
   \If{$ \| x_t \|_{M_t^{-1}} \geq \epsilon$ {\color{red} \text{and} $\|W_t-W^*\|_F\geq 1$}}{
       Draw $\tilde{y}_t \sim \text{unif}([K])$. 
   }
   \Else{
       Draw $\tilde{y}_t = \hat{y}_t \triangleq \argmax_{r\in[K]}(W_tx_t)_r$. 
   }
   \If{ $\tilde{y}_t=y_t$ }{
       $Z_t\leftarrow 1$, \\
       $M_{t+1}\leftarrow M_t + Z_t\ell_t(W_t)x_tx_t^\top$,\\
       $W_{t+1} \leftarrow \Pi_{\mathcal{W}} (W_t - \eta_{t+1}\nabla \ell_t(W_t))$, \ \ \ \ \ where $\eta_{t+1} = \frac{1}{8}$. \\
       ($\Pi_{\mathcal{W}}$ is the projection operator onto $\mathcal{W}$ w.r.t. Frobenius norm)
   }
   \Else{
       $Z_t\leftarrow 0$,  \\
       $M_{t+1}\leftarrow M_t$,\\
       $W_{t+1}\leftarrow W_t$.
   }
    
}
\end{algorithm}
\iffalse
We will pick 
\begin{align*}
     \psi(u)=\begin{cases}
        \frac{1}{2K(1-D)^2}(u-D)^2 - \frac{1}{2K}, & \text{if $u\geq 1$} \\
        \left(1+\frac{1}{K(1-D)}\right)u^2 + \left(-2-\frac{1}{K(1-D)}\right)u + 1 &\text{if $u\leq 1$},  
     \end{cases}
\end{align*}
It has the following properties: \\
(1) It is continuous and differentiable. \\
(2) It is $\frac{1}{2K(1-D)^2}$-strongly convex and $1$-smooth. \\
(3) It decreases in $[-D,D]$, passing through $(0,1),(1,0)$, and $(D,\frac{-1}{2K})$. \\
(4) Its minimum value $\frac{-1}{2K}$ is attained at $u=D$. 

\begin{lemma} 
 For the above specific choice of $\psi$, $\Phi_t$ is $\frac{1}{2KD^2}$-strongly convex and $2K$-smooth (i.e., its Hessian has eigenvalues all within $[\frac{1}{2KD^2}, 2K]$). 
 
\end{lemma}
\begin{proof}
It is clear that $\frac{\partial^2}{\partial z_j\partial z_i}\Phi_t(z)=0$ for $i\neq j$. Also, $\frac{\partial^2}{\partial^2 z_i} \Phi_t(z)\geq \frac{1}{2K(1-D)^2}\geq \frac{1}{2KD^2}$ and $\frac{\partial^2}{\partial^2 z_i} \Phi_t(z)\leq K\left(1+\frac{1}{K(1-D)}\right)\leq 2K$. 
\end{proof}

\begin{lemma}
\label{lemma:error_cond}
If $\argmax_r (Wx_t)_r = y_t$, then $\ell_t(W) \leq K$; 
If $\argmax_r (Wx_t)_r \neq y_t$, then $\ell_t(W) \geq 1$.  
\end{lemma}
\begin{proof}
If $\argmax_r (Wx_t)_r = y_t$, then $z_{y_t}-z_i\geq 0$ for all $i$, and thus $\Phi_t(z)< K\psi(0)=K$; 
If $\argmax_r (Wx_t)_r \triangleq j \neq y_t$, then $\Phi_t(z)\geq \psi(z_{y_t}-z_j) - K\times \frac{1}{2K}\geq \psi(0)-\frac{1}{2}=\frac{1}{2}$. 
This is clear by the definition of $\ell_t$. 
\end{proof}
\fi
%\begin{definition}
%$\|W\|_M^2 \triangleq \sum_{i=1}^K \|\e_i^\top W\|_M^2 $.\\
%$\gamma_\ell \triangleq \frac{1}{2KD^2}$; $\gamma_u\triangleq 2K$ (convexity and smoothity of $\Phi_t$). 
%\end{definition}


%\begin{lemma}
%For any $W^*\in \mathcal{W}$, with probability at least $1-\delta$, 
%\begin{align*} 
%    \|W_t-W^*\|_{M_t} \leq \tilde{\mathcal{O}}\left(\frac{\sqrt{Kd}}{\gamma_\ell} + D\sqrt{\gamma}\right).
%\end{align*}
%(the $\tilde{\mathcal{O}}$ notation hides $\ln T$, $\ln \frac{1}{\delta}$, $\ln \frac{1}{\gamma}$ factors)
%\end{lemma}
%\begin{proof}
%By the first order optimality condition, we have for all $W^*\in \mathcal{W}$, 
%\begin{align*}
%    \inner{\gamma W_t+\sum_{s=1}^{t-1}Z_s \nabla \ell_s(W_t), W^*-W_t}\geq 0. 
%\end{align*}
%Rearranging it and plugging in the property $\nb\ell_s(W)=\nb\Phi_s(Wx_s)x_s^\top$ gives 
%\begin{align}
%    \sum_{s=1}^{t-1} Z_s\inner{ -\nb\Phi_s(W^*x_s),W_tx_s-W^*x_s} \geq  \sum_{s=1}^{t-1} Z_s\inner{ \nb\Phi_s(W_t x_s)-\nb\Phi_s(W^*x_s),W_tx_s-W^*x_s}+\gamma\inner{W_t, W_t-W^*}. 
%    \label{eqn:key}
%\end{align}
%By $\Phi_s$'s convexity and that $\norm{W}_F^2\leq KD^2$ for all $W\in \mathcal{W}$, the RHS of \eqref{eqn:key} can be lower bounded by 
%\begin{align*}
%    \gamma_\ell\sum_{s=1}^{t-1} Z_s\norm{W_t x_s-W^*x_s}_2^2 - 2\gamma KD^2 \geq \gamma_\ell \norm{W_t-W^*}_{M_t}^2-2\gamma KD^2. 
%\end{align*}
%The LHS is upper bounded by 
%\begin{align*}
%    \sum_{i=1}^K \sum_{s=1}^{t-1} Z_s (-\nb_i\Phi_s(W^*x_s))\inner{\e_iW_t-\e_iW^*, x_s}
%\end{align*}
%\end{proof}

\begin{lemma}
\label{lemma:bound_gradient}
$\norm{\nb\ell_t(W)}_F^2 \leq 8\ell_t(W)$. 
\end{lemma}
\begin{proof}
$    \|\nb\ell_t(W)\|_F^2 = \|\nb\Phi_t(Wx_t)x_t^\top\|_{F}^2 \leq \left(2\sqrt{\Phi_t(Wx_t)}\right)^2\times 2\|x_t\|_{2}^2 \leq 8\ell_t(W).  
$
\end{proof}

\begin{lemma}
\label{lemma:decreasing error}
Let $L_{t+1}\triangleq \sum_{s=1}^t Z_s\ell_s(W_s)$. Then 
$\|W_{t+1}-W^*\|_F^2 \leq \exp\left(-\frac{L_{t+1}}{32KD^2}\right)$. 
\end{lemma}
\begin{proof}
Let $Z_{t}=1$. 
\begin{align*}
    \|W_{t+1}-W^*\|_F^2 &\leq \|W_t - \eta_{t+1}\nb\ell_t(W_t)-W^*\|_F^2 \\
    & = \| W_t - W^* \|_F^2 - 2\eta_{t+1}\inner{\nb\ell_t(W_t),W_t-W^*}_F + \eta_{t+1}^2\|\nb\ell_t(W_t)\|_F^2.  
\end{align*}
By the separable assumption we have $\ell_t(W^*)\leq 0$. Since $\ell_t$ is convex, $\inner{\nb\ell_t(W_t), W_t-W^*} \geq \ell_t(W_t)-\ell_t(W^*)\geq \ell_t(W_t)$. Continuing the above calculation and using Lemma~\ref{lemma:bound_gradient}, we get 
\begin{align*}
    \|W_{t+1}-W^*\|_F^2 &\leq \|W_t-W^*\|_2^2 - 2\eta_{t+1}\ell_t(W_t) + 8\eta_{t+1}^2 \ell_t(W_t) \\ 
    &\leq \|W_t-W^*\|_F^2 - \frac{1}{8}\ell_{t}(W_t) \\
    &\leq \|W_t-W^*\|_F^2 \left(1-\frac{\ell_t(W_t)}{32KD^2}\right)  \text{\ \ \ \ because $\|W_t-W^*\|_F^2\leq 4KD^2$} \\
    &\leq \|W_t-W^*\|_F^2 \exp\left(-\frac{\ell_t(W_t)}{32KD^2}\right)
\end{align*}
By induction, we can get
\begin{align*}
    \|W_{t+1}-W^*\|_F^2 &\leq KD^2 \exp\left( -\frac{L_{t+1}}{32KD^2} \right) 
    %&\leq KD^2 \left(\frac{8KD^2}{A_{t+1}}\right)   %\ \ \ \ \ \text{because $\exp(-x)<1/x$ for any $x>0$} \\
    %& = \frac{8K^2D^4}{A_{t+1}}. 
\end{align*}
\end{proof}

\begin{definition}
$\norm{W}_{M}^2\triangleq \sum_{i=1}^K \norm{\e_i^\top W}_{M}^2$.
\end{definition}
With this definition we have $\|Wx_t\|_2^2 = \sum_{i=1}^K (\e_i^\top Wx_t)^2 \leq \sum_{i=1}^K \|\e_i^\top W\|_M^2 \|x_t\|_{M^{-1}}^2\leq \|W\|_M^2 \|x_t\|_{M^{-1}}^2$

\begin{lemma}
\begin{align*}
\|W_{t}-W^*\|_{M_{t}}^2 \leq (1+L_t)K^2D^2 \exp\left(-\frac{L_t}{32KD^2}\right) \leq 32K^3D^4. 
\end{align*}
\end{lemma}
\begin{proof}
Because we assume $\|x_t\|_2^2\leq 1$, it holds that $ M_t \preceq (1+L_t) I$. Therefore $\|W_t-W^*\|_{M_t}^2 \leq (1+L_t)\|W_t-W^*\|_I^2 =  (1+L_t) \sum_{i=1}^K \| \e_i^\top (W_t-W^*) \|_2^2 \leq (1+L_t) K \|W_t-W^*\|_F^2 $. By Lemma~\ref{lemma:decreasing error} this is bounded by $(1+L_t)K^2D^2\exp\left( -\frac{L_t}{32KD^2} \right)$, which can further be bounded by a constant related to $K$ and $D$. For example, using the property $\exp(-x)\leq \frac{1}{(1+x)^2}$ for all $x>0$, it can be upper bounded by $(1+L_t)K^2D^2 \times \frac{(32KD^2)^2}{(L_t+32KD^2)^2} \leq \frac{32^2K^4D^6}{32KD^2+L_t}\leq 32K^3D^4$.   
\end{proof}

\begin{lemma}
\label{lemma: xbounded}
If $\|x_t\|_{M_t^{-1}}\leq \epsilon = \frac{1}{4D\sqrt{32K^3D^4}}$, then $\hat{y}_t=y_t$. 
\end{lemma}
\begin{proof}
By the convexity of $\ell_t$, 
\begin{align*}
    \ell_t(W_t)\leq \ell_t(W_t)-\ell_t(W^*)&\leq \inner{\nb \ell_t(W_t), W_t-W^*}\\
     & = \inner{\nb\Phi_t(W_tx_t)x_t^\top, W_t - W^*} \\
    & = \inner{\nb\Phi_t(W_tx_t), W_tx_t - W^*x_t} \\
    & \leq 4D\norm{W_tx_t-W^*x_t}_2 \\
    & \leq 4D\norm{W_t-W^*}_{M_t}\norm{x_t}_{M_t^{-1}} \leq 1.
\end{align*}
This implies $\hat{y}_t=y_t$. 
\end{proof}

Therefore, when we do not explore, we know $W_t$ will predict correctly! Thus we only need to bound the number of errors occurred in exploration rounds, which is calculated by the following lemma. 
\begin{lemma}
\label{lemma:N}
$\sum_{t=1}^T \one[\tilde{y}_t\neq y_t] \leq ???$ with probability at least $1-\delta$. 
\end{lemma}

\begin{proof}
By the above discussion, $\sum_{t=1}^T \one[\tilde{y}_t\neq y_t] \leq N\triangleq \sum_{t=1}^T Z_t$, the number of exploration rounds. 
\begin{align*}
    N &= \sum_{t=1}^T \one\left[\norm{x_t}_{M_t^{-1}}> \epsilon\right]\\
    &\leq \left(K\ln \frac{1}{\delta}\right)\sum_{t=1}^T \one\left[\norm{x_t}_{M_t^{-1}}> \epsilon\right]Z_t \tag{when $\|x_t\|_{M_t^{-1}}\geq \epsilon$, $\tilde{y}_t=y_t$ with probability $\frac{1}{K}$} \\
    & \leq \frac{K\ln \frac{1}{\delta}}{\epsilon^2}\sum_{t=1}^T \norm{x_t}_{M_t^{-1}}^2 Z_t \leq \max_{t\in[T]}\left(\frac{1}{\ell_{t}(W_t)}\right)\times \frac{K\ln T\ln \frac{1}{\delta}}{\epsilon^2}. 
\end{align*}
\end{proof}


\textbf{Discussion}. 
In the calculation of Lemma~\ref{lemma: xbounded},  we can actually get $\ell_t(W_t)^2 \leq \norm{\nb\Phi_t(W_tx_t)}_2^2\norm{W_t-W^*}_{M_t}^2\norm{x_t}_{M_t^{-1}}^2$. Similar to the calculation in Lemma~\ref{lemma:decreasing error}, $\norm{\nb\Phi_t(W_tx_t)}_2^2$ is bounded by constant times $\ell_t(W_t)$. So the exploration criterion could potentially become $\ell_t(W_t)\|x_t\|_{M_t^{-1}}^2\geq \frac{1}{\epsilon^2}$, which makes Lemma~\ref{lemma:N} go through. The problem is just we do not know $\ell_t(W_t)$ in general. 


%\section{Reservoir Sampling}
%\textbf{Observation}. In Algorithm 2 of the SOBA paper, $\tilde{\ell}$ is constructed as a $ \tilde{\mathcal{O}}(\gamma \eta)$-exp-concave function which appears at the rate of $\gamma$. What if we divide a $\tilde{\mathcal{O}}(\gamma\eta)$-exp-concave function into roughly $1/\gamma$ $\tilde{\mathcal{O}}(\eta)$-exp-concave functions?




%\begin{algorithm}[H]
%\caption{Banditron}
%\textbf{Input}: $D\geq 2$, $\epsilon$, $\gamma$. \\
%\textbf{Definition}: 

%\end{algorithm}







\section{Continuous EXP4 with Uniform Exploration}

\begin{algorithm}[H]
\caption{Banditron}
\textbf{Parameters}: feasible set $\Omega\subset \mathbb{R}^{K\times d}$ \\
\textbf{Definitions}: $\ell_t(W)\triangleq [1-(Wx_t)_{y_t}+\max_{r\in[K]}(Wx_t)_r]_+$ (hinge loss)\\
\For{$t=1, \ldots, T$}{
    Receive $x_t\in \mathbb{R}^d$. \\
    Define 
    \begin{gather*}
    q_t(W)=\frac{\exp(-\alpha \sum_{s=1}^{t-1} \hat{\ell}_s(W))}{\int_{U\in \Omega} \exp(-\alpha \sum_{s=1}^{t-1} \hat{\ell}_s(U))dU }, \ \ \ \ \forall W\in\Omega, 
    %\Omega_k^t\triangleq\{W\in\Omega: k=\argmax_{r\in[K]} (Wx_t)_r\} \ \ \ \text{i.e., the set of $W$'s that predict class $k$ at time $t$}\\
    %p_t(k)=\int_{W\in \Omega_k^t} q_t(W)dW, \ \ \ \ \forall k\in[K], \\
    %\tilde{p}_t(k) \propto p_t(k)\one[p_t(k)\geq \epsilon]
    \end{gather*} 
    where $\hat{\ell}_s(W)=\one[\tilde{y}_s=y_s]\left(\frac{\one[\hat{y}_s= y_s]\ell_s(W)}{1-\gamma+\frac{\gamma}{K}} + \frac{\one[\hat{y}_s\neq y_s]\ell_s(W)}{\frac{\gamma}{K}}\right)$.\\
    Sample $W_t\sim q_t$, and let $\hat{y}_t = \argmax_{r\in[K]}(W_tx_t)_r$.\\
    Let $\tilde{y}_t=\hat{y}_t$ with probability $1-\gamma$, and $\tilde{y}_t\sim \text{unif}([K])$ with probability $\gamma$. 
}
\end{algorithm}

%\begin{assumption}
%\label{assump:zero}
%   Assume we know there exists a $W^*\in \Omega$ having zero loss (i.e., $\ell_t(W^*)=0$ for all $t$). 
%\end{assumption}

%\begin{definition}
%Define $\tilde{q}_t(W) = q_t(W)\times \frac{\tilde{p}_t(k')}{p_t(k')}$ with $k' = \argmax_r(Wx_t)_r$ (i.e., letting $\tilde{q}_t$ scales with $\tilde{p}_t$). 
%\end{definition}
%\begin{definition}
%Define $\bar{\ell}_t(k)=\int_{W\in\Omega_k^t} \tilde{q}_t(W)\ell_t(W)dW \big/ \int_{W\in\Omega_k^t} \tilde{q}_t(W)dW $, that is, the average loss over $W$'s that predicts class $k$. Defining this is because our problem is slightly different from traditional contextual bandit formulation: with hinge loss, the loss received by different experts may be different even if they predict the same class/action. Note that while $\ell_t(\cdot), \hat{\ell}_t(\cdot)$ are losses for experts, $\bar{\ell}_t(\cdot)$ is loss for actions. 
%\end{definition}

%\begin{lemma}
%$\int_{W\in\Omega}q_t(W)\hat{\ell}_t(W)\geq (1-K\epsilon)\bar{\ell}_t(\hat{y}_t)$. 
%\end{lemma}
%\begin{proof}
%By the definition of $\tilde{p}_t$, we have $1\leq \frac{\tilde{p}_t(k)}{p_t(k)}\leq \frac{1}{1-K\epsilon}$ for all $k$. Thus $1\leq \frac{\tilde{q}_t(W)}{q_t(W)}\leq \frac{1}{1-K\epsilon}$ for all $W$.  Therefore, 
%\begin{align*}
%    \int_{W\in\Omega}q_t(W)\hat{\ell}_t(W)dW&\geq (1-K\epsilon)\int_{W\in\Omega}\tilde{q}_t(W)\hat{\ell}_t(W)dW \\
%    &=(1-K\epsilon)\left(\one[\hat{y}_t=y_t]\sum_{k=1}^K \tilde{p}_t(k)\bar{\ell}_t(k) + \one[\hat{y}_t\neq y_t]\sum_{k=1}^K \tilde{p}_t(k)\bar{a}_t(k)\right). 
%\end{align*}
%\end{proof}

%\begin{lemma}
%\label{lemma:simple}
%The following facts hold: \\
%(1) $ \hat{\ell}_t(W)\leq \ell_t(W)$ for all $W$. \\
%(2) $ \hat{\ell}_t(W_t) \geq \ell_t(W_t) $. 
%\end{lemma}
%\begin{proof}
%\ \\
%(1) $ \hat{\ell}_t(W)\leq \one[\hat{y}_t=y_t]\ell_t(W)+\one[\hat{y}_t\neq y_t]\ell_t(W) = \ell_t(W)$. \\
%(2) $ \hat{\ell}_t(W_t)\geq \one[\hat{y}_t\neq y_t]a_t(W_t)=\one[\hat{y}_t\neq y_t]=\ell_t(W_t)$. 
%\end{proof}

%By standard analysis of the exponential weight algorithm (or see page 9 of \cite{bubeck2017kernel}), we have 
%\begin{align*}
%\E\left[\sum_{t=1}^T\int_{W\in\Omega} q_t(W)\hat{\ell}_t(W)dW \right] &\leq \E\left[\sum_{t=1}^T\hat{\ell}_t(W^*) + \frac{\textbf{Ent}(q_1||\delta(W^*))}{\alpha} + \alpha \sum_{t=1}^T\int_{W\in\Omega} q_t(W)\hat{\ell}_t(W)^2dW\right] \\ 
%&\leq \E\left[\sum_{t=1}^T\hat{\ell}_t(W^*) + \frac{\textbf{Ent}(q_1||\delta(W^*))}{\alpha} + \alpha \sum_{t=1}^T\int_{W\in\Omega} q_t(W)\hat{\ell}_t(W)dW \right],
%\end{align*}
%which implies $\E\left[\sum_{t=1}^T\hat{\ell}_t(W_t)\right]\leq  \E\left[\sum_{t=1}^T  \hat{\ell}_t(W^*) \right]+\frac{\textbf{Ent}(q_1||\delta(W^*))}{\alpha(1-\alpha)}+\frac{\alpha}{1-\alpha}\E\left[\sum_{t=1}^T\hat{\ell}_t(W^*)\right] $. Combining Lemma~\ref{lemma:simple}, we get a small-loss regret bound. 

%\textbf{Note.} Considering $\delta(W^*)$ as a distribution in a region surrounding $W^*$ which has $1/(TD)^{dK}$ $\Omega$'s volume, then $\textbf{Ent}(q_1||\delta(W^*))$ should be in the order of $dK\log (TD)$. 

%\textbf{Discussion.} \\
%(1) It seems even okay to define $\hat{\ell}_s(W)=\one[\hat{y}_s\neq y_s]a_s(W)$. \\
%(2) $q_t(\cdot)$ can be sampled when $\sum_{s=1}^{t-1}\hat{\ell}_s(W)$ is convex (not very sure, but I believe so from papers like \cite{lovasz2006fast,narayanan2013efficient,bubeck2015sampling}). The part $ \one[\hat{y}_s=y_s]\ell_s(W)$ can become convex if we define $\ell(W)$ to be the hinge loss; however, the part $\one[\hat{y}_s\neq y_s]a_s(W)$ does not seem to be easy. This corresponds to constructing a loss function when the learner is wrong, and is somewhat related to the issue we met previously: the region $\{W: \hat{y}_t\neq \argmax_r (Wx_t)_r \}$ is non-convex. 


%\section{Continuous EXP4 for Banditron with Uniform Exploration}
\begin{lemma}
$\E_{\tilde{y}_t}[\hat{\ell}_t(W)] = \ell_t(W)$ for all $W$. 
\end{lemma}
\begin{proof}
\begin{align*}
\E_{\tilde{y}_t}[\hat{\ell}_t(W)] &= \E_{\tilde{y}_t}\left[ \one[\hat{y}_s=y_s] \frac{\one[\tilde{y}_s=y_s]\ell_t(W)}{1-\gamma+\frac{\gamma}{K}}   +
\one[\hat{y}_s\neq y_s] \frac{\one[\tilde{y}_s = y_s]\ell_t(W)}{\frac{\gamma}{K}}\right] \\
& = \one[\hat{y}_s=y_s] \ell_t(W)   +
\one[\hat{y}_s\neq y_s] \ell_t(W) = \ell_t(W). 
\end{align*}
\end{proof}

%\begin{lemma}
%$\E_{\tilde{y}_t}[(\hat{\ell}_t(W))^2] \leq \frac{2K}{\gamma}(\ell_t(W))^2\leq \frac{2K}{\gamma}\ell_t(W)$ for all $W$. 
%\end{lemma}

Plugging these lemmas in the previous hedge bound, we can get \begin{align*}
&\E\left[\sum_{t=1}^T  \ell_{t}(W_{t})  \right]\\
&=\E\left[\sum_{t=1}^T  \int_{W\in\Omega} q_t(W)\ell_t(W)dW  \right] \\
&=\E\left[\sum_{t=1}^T \E_{\tilde{y}_t} \left[ \int_{W\in\Omega} q_t(W)\hat{\ell}_t(W)dW \right] \right]\\
&\leq \E\left[\sum_{t=1}^T \E_{\tilde{y}_t} \left[ \hat{\ell}_t(W^*) \right]+ \frac{\textbf{Ent}(q_1||\delta(W^*))}{\alpha} + \alpha \sum_{t=1}^T \E_{\tilde{y}_t}\left[ \int_{W\in\Omega} q_t(W)\hat{\ell}_t(W)^2dW \right] \right] \\ 
&\leq \E\left[\sum_{t=1}^T \E_{\tilde{y}_t}\left[ \hat{\ell}_t(W^*) \right] + \frac{\textbf{Ent}(q_1||\delta(W^*))}{\alpha} + \frac{2K\alpha}{\gamma} \sum_{t=1}^T \E_{\tilde{y}_t}\left[ \int_{W\in\Omega} q_t(W)\hat{\ell}_t(W)dW \right]\right]
\end{align*}

... to bound the regret, it would be something like bounding $\frac{1}{1-\frac{K\alpha}{\gamma}}\left(\frac{1}{\alpha}+\frac{K\alpha}{\gamma}L^* + \gamma T\right)$, which gives $(L^*T)^{1/3} + \sqrt{T}$ regret bound. 


\textbf{Discussion.} We can change $\ell_t(\cdot)$ to any reasonable convex loss (e.g., logsitic loss or second-order loss). 





\newpage
%\bibliography{adaptivity} 
%\bibliographystyle{plain}

\end{document} 
